\chapter{Static Database Access}

\section{Overview}

An IOC database is created on a Unix system via a Database Configuration Tool and stored in a Unix file.
EPICS provides two sets of database access routines:
Static Database Access and Runtime Database Access.
Static database access can be used on Unix or IOC database files.
Runtime database requires an initialized IOC database.
Static database access is described in this chapter, runtime database access in the next chapter.

Static database access provides a simplified interface to a database, i.e. much of the complexity is hidden.
\verb|DBF_MENU| and \verb|DBF_DEVICE| fields are accessed via a common type called \verb|DCT_MENU|.
A set of routines are provided to simplify access to link fields.
All fields can be accessed as character strings.
This interface is called static database access because it can be used to access an uninitialized as well as an initialized database.

Before accessing database records, the menus, record types, and devices used to define that IOC database must be read via \verb|dbReadDatabase| or \verb|dbReadDatabaseFP|.
These routines, which are also used to load record instances, can be called multiple times.

Database Configuration Tools (DCTs) should manipulate an EPICS database only via the static database access interface.
An IOC database is created on a host system via a database configuration tool and stored in a host file with a file extension of ``\verb|.db|".
Three routines (\verb|dbReadDatabase|, \verb|dbReadDatabaseFP| and \verb|dbWriteRecord|) access the database file.
These routines read/write a database file to/from a memory resident EPICS database.
All other access routines manipulate the memory resident database.

An include file \verb|dbStaticLib.h| contains all the definitions needed to use the static database access library.
Two structures (\verb|DBBASE| and \verb|DBENTRY|) are used to access a database.
The fields in these structures should not be accessed directly.
They are used by the static database access library to keep state information for the caller.

\section{Definitions}

\subsection{DBBASE}

Multiple memory resident databases can be accessed simultaneously.
The user must provide definitions in the form:

\begin{lstlisting}[language=C]
DBBASE *pdbbase;
\end{lstlisting}

NOTE: On an IOC \index{pdbbase}pdbbase is a global variable, which is accessable if you include dbAccess.h

\subsection{DBENTRY}

A typical declaration for a database entry structure is:

\begin{lstlisting}[language=C]
DBENTRY *pdbentry;
pdbentry=dbAllocEntry(pdbbase);
\end{lstlisting}

Most static access to a database is via a \verb|DBENTRY| structure.
As many \verb|DBENTRYs| as desired can be allocated.

The user should NEVER access the fields of \verb|DBENTRY| directly.
They are meant to be used by the static database access library.

Most static access routines accept an argument which contains the address of a \verb|DBENTRY|.
Each routine uses this structure to locate the information it needs and gives values to as many fields in this structure as possible.
All other fields are set to \verb|NULL|.

\subsection{Field Types}
\label{subsec:Field Types}

\index{Database Configuration Field Types}
Each database field has a type as defined in the next chapter.
For static database access a simpler set of field types are defined.
In addition, at runtime, a database field can be an array.
With static database access, however, all fields are scalars.
Static database access field types are called DCT field types.

The DCT field types are:

\begin{itemize}
\item \index{DCT\_STRING}\verb|DCT_STRING|: Character string.

\item \index{DCT\_INTEGER}\verb|DCT_INTEGER|: Integer value

\item \index{DCT\_REAL}\verb|DCT_REAL|: Floating point number

\item \index{DCT\_MENU}\verb|DCT_MENU|: A set of choice strings

\item \index{DCT\_MENUFORM}\verb|DCT_MENUFORM|: A set of choice strings with associated form.

\item \index{DCT\_INLINK}\verb|DCT_INLINK|: Input Link

\item \index{DCT\_OUTLINK}\verb|DCT_OUTLINK|: Output Link

\item \index{DCT\_FWDLINK}\verb|DCT_FWDLINK|: Forward Link

\item \index{DCT\_NOACCESS}\verb|DCT_NOACCESS|: A private field for use by record access routines

\end{itemize}

A \verb|DCT_STRING| field contains the address of a \verb|NULL| terminated string.
The field types \verb|DCT_INTEGER| and \verb|DCT_REAL| are used for numeric fields.
A field that has any of these types can be accessed via the \verb|dbGetString|, \verb|dbPutString|, \verb|dbVerify|, and \verb|dbGetRange| routines.

The field type \verb|DCT_MENU| has an associated set of strings defining the choices.
Routines are available for accessing menu fields.
A menu field can also be accessed via the \verb|dbGetString|, \verb|dbPutString|, \verb|dbVerify|, and \verb|dbGetRange| 
routines.

The field type \verb|DCT_MENUFORM| is like \verb|DCT_MENU| but in addition the field has an associated link field.

\verb|DCT_INLINK| (input), \verb|DCT_OUTLINK| (output), and \verb|DCT_FWDLINK| (forward) specify that the field is a link, which has an associated set of static access routines described in the next subsection.
A field that has any of these types can also be accessed via the \verb|dbGetString|, \verb|dbPutString|, \verb|dbVerify|, and \verb|dbGetRange| routines.

\section{Allocating and Freeing DBBASE}

\subsection{dbAllocBase}

\index{dbAllocBase}
\begin{lstlisting}[language=C]
DBBASE *dbAllocBase(void);
\end{lstlisting}

This routine allocates and initializes a DBBASE structure.
It does not return if it is unable to allocate storage.

Most applications should not need to call this routine directly.
The \verb|dbReadDatabase| and \verb|dbReadDatabaseFP| routines will call it automatically if \verb|pdbbase| is null.
Thus an application normally only has to contain code like the following:

\begin{lstlisting}[language=C]
DBBASE  *pdbbase=0;
...
status = dbReadDatabase(&pdbbase, dbdfile, search_path, macros);
\end{lstlisting}

However the static database access library does allow applications to work with multiple databases simultaneously, each referenced via a different DBBASE pointer.
Such applications may need to call \verb|dbAllocBase| directly.

\subsection{dbFreeBase}

\index{dbFreeBase}
\begin{lstlisting}[language=C]
void dbFreeBase(DBBASE *pdbbase);
\end{lstlisting}

\verb|dbFreeBase| frees the entire database reference by \verb|pdbbase| including the DBBASE structure itself.

\section{DBENTRY Routines}

\subsection{Alloc/Free DBENTRY}

\index{Alloc/Free DBENTRY}
\begin{lstlisting}[language=C]
DBENTRY *dbAllocEntry(DBBASE *pdbbase);
void dbFreeEntry(DBENTRY *pdbentry);
\end{lstlisting}

\index{dbAllocEntry}
\index{dbFreeEntry}
These routines allocate, initialize, and free \verb|DBENTRY| structures. The user can allocate and free \verb|DBENTRY| structures as 
necessary. Each \verb|DBENTRY| is, however, tied to a particular database.

\verb|dbAllocEntry| and \verb|dbFreeEntry| act as a pair, i.e. the user calls \verb|dbAllocEntry| to create a new DBENTRY and 
calls \verb|dbFreeEntry| when done.

\subsection{dbInitEntry dbFinishEntry}

\index{dbInitEntry}
\index{dbFinishEntry}
\begin{lstlisting}[language=C]
void dbInitEntry(DBBASE *pdbbase,DBENTRY *pdbentry);
void dbFinishEntry(DBENTRY *pdbentry);
\end{lstlisting}

\index{dbInitEntry}
\index{dbFinishEntry}
The routines \verb|dbInitEntry| and \verb|dbFinishEntry| are provided in case the user wants to allocate a \verb|DBENTRY| structure 
on the stack. Note that the caller MUST call \verb|dbFinishEntry| before returning from the routine that calls 
\verb|dbInitEntry|. An example of how to use these routines is:

\begin{lstlisting}[language=C]
int xxx(DBBASE *pdbbase)
{
    DBENTRY dbentry;
    DBENTRY *pdbentry = &dbentry;
    ...
    dbInitEntry(pdbbase,pdbentry);
    ...
    dbFinishEntry(pdbentry);
}

\end{lstlisting}

\subsection{dbCopyEntry}

dbCopyEntry

Contents

\begin{lstlisting}[language=C]
DBENTRY *dbCopyEntry(DBENTRY *pdbentry);
void dbCopyEntryContents(DBENTRY *pfrom,DBENTRY *pto);
\end{lstlisting}

\index{dbCopyEntry}
\index{dbCopyEntryContents}
The routine \verb|dbCopyEntry| allocates a new entry, via a call to \verb|dbAllocEntry|, copies the information from the original 
entry, and returns the result. The caller must free the entry, via \verb|dbFreeEntry| when finished with the DBENTRY.

The routine \verb|dbCopyEntryContents| copies the contents of pfrom to pto. Code should never perform structure copies.

\section{Read and Write Database}

\subsection{Read Database File}

\begin{lstlisting}[language=C]
long dbReadDatabase(DBBASE **ppdbbase,const char *filename,
    char *path, char *substitutions);
long dbReadDatabaseFP(DBBASE **ppdbbase,FILE *fp,
    char *path, char *substitutions);
long dbPath(DBBASE *pdbbase,const char *path);
long dbAddPath(DBBASE *pdbbase,const char *path);
\end{lstlisting}

\index{dbReadDatabase}
\verb|dbReadDatabase| and \index{dbReadDatabaseFP}\verb|dbReadDatabaseFP| both read a file containing database definitions as described in chapter 
"Database Definitions". If \verb|*ppdbbase| is NULL, \verb|dbAllocBase| is automatically invoked and the return address 
assigned to \verb|*pdbbase|. The only difference between the two routines is that one accepts a file name and the other a 
"FILE *". Any combination of these routines can be called multiple times. Each adds definitions with the rules described 
in chapter ``Database Definitions".

The routines \index{dbPath}\verb|dbPath| and \index{dbAddPath}\verb|dbAddPath| specify paths for use by include statements in database definition files. These are 
not normally called by user code.

\subsection{Write Database Definitons}

\begin{lstlisting}[language=C]
long dbWriteMenu(DBBASE *pdbbase,char *filename,char *menuName);
long dbWriteMenuFP(DBBASE *pdbbase,FILE *fp,char *menuName);
long dbWriteRecordType(DBBASE *pdbbase,char *filename,char *recordTypeName);
long dbWriteRecordTypeFP(DBBASE *pdbbase,FILE *fp,char *recordTypeName);
long dbWriteDevice(DBBASE *pdbbase,char *filename);
long dbWriteDeviceFP(DBBASE *pdbbase,FILE *fp);
long dbWriteDriver(DBBASE *pdbbase,char *filename);
long dbWriteDriverFP(DBBASE *pdbbase,FILE *fp);
long dbWriteRegistrarFP(DBBASE *pdbbase,FILE *fp);
long dbWriteFunctionFP(DBBASE *pdbbase,FILE *fp);
long dbWriteVariableFP(DBBASE *pdbbase,FILE *fp);
long dbWriteBreaktable(DBBASE *pdbbase,const char *filename);
long dbWriteBreaktableFP(DBBASE *pdbbase,FILE *fp);
\end{lstlisting}

Each of these routines writes files in the same format accepted by \verb|dbReadDatabase| and \verb|dbReadDatabaseFP|.
Two versions of each type are provided.
The only difference is that one accepts a filename string and the other a \verb|FILE *| pointer.
Thus only one of each type will be described.

\index{dbWriteMenu}
\index{dbWriteMenuFP}
\verb|dbWriteMenu| writes the description of the specified menu or, if \verb|menuName| is NULL, the descriptions of all menus.

\index{dbWriteRecordType}
\index{dbWriteRecordTypeFP}
\verb|dbWriteRecordType| writes the description of the specified record type or, if \verb|recordTypeName| is NULL, the 
descriptions of all record types to the named file.

\index{dbWriteDevice}
\index{dbWriteDeviceFP}
\verb|dbWriteDevice| writes the description of all devices to the named file.

\index{dbWriteDriver}
\index{dbWriteDriverFP}
\verb|dbWriteDriver| writes the description of all drivers to the named file.

\index{dbWriteRegistrarFP}
\verb|dbWriteRegistrarFP| writes the list of all registrars to the given open file (no filename version is provided).

\index{dbWriteFunctionFP}
\verb|dbWriteFunctionFP| writes the list of all functions to the given open file (no filename version is provided).

\index{dbWriteVariableFP}
\verb|dbWriteVariableFP| writes the list of all variables to the given open file (no filename version is provided).

\index{dbWriteBreaktable}
\index{dbWriteBreaktableFP}
\verb|dbWriteBreaktable| writes the definitions of all breakpoint tables to the named file.

\subsection{Write Record Instances}

\begin{lstlisting}[language=C]
long dbWriteRecord(DBBASE *pdbbase,char * file,

char *precordTypeName,int level);
long dbWriteRecordFP(DBBASE *pdbbase,FILE *fp,

char *precordTypeName,int level);
\end{lstlisting}

\index{dbWriteRecord}
\index{dbWriteRecordFP}
These routines write record instance data.
If \verb|precordTypeName| is NULL, then the record instances for all record types are written, otherwise only the records for the specified type are written.
\verb|level| has the following meaning:

\begin{itemize}
\item 0 - Write only prompt fields that are different than the default value.
\item 1 - Write only the fields which are prompt fields.
\item 2 - Write the values of all fields.
\end{itemize}

\section{Manipulating Record Types}

\subsection{Get Number of Record Types}

\begin{lstlisting}[language=C]
int  dbGetNRecordTypes(DBENTRY *pdbentry);
\end{lstlisting}

\index{dbGetNRecordTypes}
This routine returns the number of record types in the database.

\subsection{Locate Record Type}

\begin{lstlisting}[language=C]
long dbFindRecordType(DBENTRY *pdbentry,
char *recordTypeName);
long dbFirstRecordType(DBENTRY *pdbentry);
long dbNextRecordType(DBENTRY *pdbentry);
\end{lstlisting}

\index{dbFindRecordType}
\index{dbFirstRecordType}
\index{dbNextRecordType}
\verb|dbFindRecordType| locates a particular record type. \verb|dbFirstRecordType| locates the first, in alphabetical order, 
record type. Given that DBENTRY points to a particular record type, \verb|dbNextRecordType| locates the next record type. 
Each routine returns 0 for success and a non zero status value for failure. A typical code segment using these routines is:

\begin{lstlisting}[language=C]
status = dbFirstRecordType(pdbentry);
while(!status) {
    /*Do something*/
    status = dbNextRecordType(pdbentry)
}
\end{lstlisting}

\subsection{Get Record Type Name}

\begin{lstlisting}[language=C]
char *dbGetRecordTypeName(DBENTRY *pdbentry);
\end{lstlisting}

\index{dbGetRecordTypeName}
This routine returns the name of the record type that DBENTRY currently references. This routine should only be called 
after a successful call to \verb|dbFindRecordType|, \verb|dbFirstRecordType|, or \verb|dbNextRecordType|. It returns NULL if 
DBENTRY does not point to a record description.

\section{Manipulating Field Descriptions}

The routines described here all assume that DBENTRY references a record type, i.e. that \verb|dbFindRecordType|, \verb|dbFirstRecordType| or \verb|dbNextRecordType| have returned success or that a record instance has been successfully located.

\subsection{Get Number of Fields}

\begin{lstlisting}[language=C]
int  dbGetNFields(DBENTRY *pdbentry,int dctonly);
\end{lstlisting}

\index{dbGetNFields}
Returns the number of fields for the record instance that DBENTRY currently references.

\subsection{Locate Field}

\begin{lstlisting}[language=C]
long dbFirstField(DBENTRY *pdbentry,int dctonly);
long dbNextField(DBENTRY *pdbentry,int dctonly);
\end{lstlisting}

\index{dbFirstField}
\index{dbNextField}
\index{dbFoundField}
These routines are used to locate fields. If any of these routines returns success, then DBENTRY references that field 
description. 

\subsection{Get Field Type}

\begin{lstlisting}[language=C]
int  dbGetFieldType(DBENTRY *pdbentry);
\end{lstlisting}

\index{dbGetFieldType}
This routine returns the integer value for a DCT field type.
See Section \ref{subsec:Field Types} for a description of the field types.

\subsection{Get Field Name}

\begin{lstlisting}[language=C]
char *dbGetFieldName(DBENTRY *pdbentry);
\end{lstlisting}

\index{dbGetFieldName}
This routine returns the name of the field that DBENTRY currently references. It returns NULL if DBENTRY does not 
point to a field.

\subsection{Get Default Value}

\begin{lstlisting}[language=C]
char *dbGetDefault(DBENTRY *pdbentry);
\end{lstlisting}

\index{dbGetDefaultName}
This routine returns the default value for the field that DBENTRY currently references. It returns NULL if DBENTRY 
does not point to a field or if the default value is NULL.

\subsection{Get Field Prompt}
\label{subsec:Get Field Prompt}

\begin{lstlisting}[language=C]
char *dbGetPrompt(DBENTRY *pdbentry);
int   dbGetPromptGroup(DBENTRY *pdbentry);
char *dbGetPromptGroupNameFromKey(DBBASE *pdbbase, const short key);
short dbGetPromptGroupKeyFromName(DBBASE *pdbbase, const char *name);
\end{lstlisting}

\index{dbGetPrompt}
\index{dbGetPromptGroup}
\index{dbGetPromptGroupNameFromKey}
\index{dbGetPromptGroupKeyFromName}
The \verb|dbGetPrompt| routine returns the character string prompt value, which provides a short description of the field.
\verb|dbGetPromptGroup| returns the field's group key.

Conversion between the group key and the group name as a string is provided by two functions:
\verb|dbGetPromptGroupNameFromKey| returns a pointer to a static string containing the name of the group,
\verb|NULL| for an invalid key.
\verb|dbGetPromptGroupKeyFromName| returns the numerical key related to the specified group name string,
\verb|0| if the string does not match an existing group name.

\section{Manipulating Record Attributes}

A record attribute is a psuedo-field definition attached to a record type. If a attribute value is assigned to a psuedo field 
name then all record instances of that record type appear to have that field with the defined value. All attribute fields are 
\verb|DCT_STRING| fields.

Two field attributes are automatically created: RTYP and VERS. RTYP is set equal to ,the record type name. VERS is 
initialized to the value ``none specified" but can be changed by record support.

\subsection{dbPutRecordAttribute}

\begin{lstlisting}[language=C]
long dbPutRecordAttribute(DBENTRY *pdbentry, const char *name,
    const char *value);
\end{lstlisting}

\index{dbPutRecordAttribute}
This creates or modifies the attribute \emph{name} of the record type referenced by \emph{pdbentry} to \emph{value}.
Attribute names should be valid C identifiers, starting with a letter or underscore followed by any number of alphanumeric or underscore characters.

\subsection{dbGetRecordAttribute}

\begin{lstlisting}[language=C]
long dbGetRecordAttribute(DBENTRY *pdbentry, const char *name);
\end{lstlisting}

\index{dbGetRecordAttribute}
Looks up the attribute \emph{name} of the record type referenced by \emph{pdbentry} and sets the the field pointer in \emph{pdbentry} to refer to this string if it exists.
The routine \verb|dbGetString| can be used subsequently to read the current attribute value.

\section{Manipulating Record Instances}

With the exception of \verb|dbFindRecord|, each of the routines described in this section need DBENTRY to reference a valid record type, i.e. that \verb|dbFindRecordType|, \verb|dbFirstRecordType|, or \verb|dbNextRecordType| have been called and returned success.

\subsection{Get Number of Records}

\begin{lstlisting}[language=C]
int  dbGetNRecords(DBENTRY *pdbentry);
\end{lstlisting}

\index{dbGetNRecords}
Returns the total number of record instances and aliases for the record type that DBENTRY currently references.

\subsection{Get Number of Record Aliases}

\begin{lstlisting}[language=C]
int dbGetNAliases(DBENTRY *pdbentry)
\end{lstlisting}

\index{dbGetNAliases}
Returns the number of record aliases for the record type that DBENTRY currently references.

\subsection{Locate Record}

\begin{lstlisting}[language=C]
long dbFindRecord(DBENTRY *pdbentry,char *precordName);
long dbFirstRecord(DBENTRY *pdbentry);
long dbNextRecord(DBENTRY *pdbentry);
\end{lstlisting}

\index{dbFindRecord}
\index{dbFirstRecord}
\index{dbNextRecord}
These routines are used to locate record instances and aliases.
If any of these routines returns success, then DBENTRY references a record or a record alias (use \verb|dbIsAlias| to distinguish the two).
\verb|dbFindRecord| may be called without DBENTRY referencing a valid record type.
\verb|dbFirstRecord| only works if DBENTRY references a record type.
The \verb|dbDumpRecords| example given at the end of this chapter shows how these routines can be used.

\verb|dbFindRecord| also calls \verb|dbFindField| if the record name includes a field name, i.e. it ends in ``\verb|.XXX|".
The routine \verb|dbFoundField| indicates whether the field was found or not.
If it was not found, then \verb|dbFindField| must be called before individual fields can be accessed.

\subsection{Get Record Name}

\begin{lstlisting}[language=C]
char *dbGetRecordName(DBENTRY *pdbentry);
\end{lstlisting}

\index{dbGetRecordName}
This routine only works properly if called after \verb|dbFindRecord|, \verb|dbFirstRecord|, or \verb|dbNextRecord| has returned success.
If DBENTRY refers to an alias, the name returned is that of the alias, not of the record it refers to.

\subsection{Distinguishing Record Aliases}

\begin{lstlisting}[language=C]
int dbIsAlias(DBENTRY *pdbentry)
\end{lstlisting}

\index{dbIsAlias}
This routine only works properly if called after \verb|dbFindRecord|, \verb|dbFirstRecord|, or \verb|dbNextRecord| has returned success.
If DBENTRY refers to an alias it returns a non-zero value, otherwise it returns zero.

\subsection{Create/Delete/Free Records and Aliases}

\begin{lstlisting}[language=C]
long dbCreateRecord(DBENTRY *pdbentry,char *precordName);
long dbCreateAlias(DBENTRY *pdbentry, const char *paliasName);
long dbDeleteRecord(DBENTRY *pdbentry);
long dbDeleteAliases(DBENTRY *pdbentry);
long dbFreeRecords(DBBASE *pdbbase);
\end{lstlisting}

\index{dbCreateRecord}
\index{dbCreateAlias}
\index{dbDeleteRecord}
\index{dbDeleteAliases}
\index{dbFreeRecords}
\verb|dbCreateRecord|, which assumes that \verb|DBENTRY| references a valid record type, creates a new record instance and 
initializes it as specified by the record description.
If it returns success, then \verb|DBENTRY| references the record just created.
\verb|dbCreateAlias| assumes that DBENTRY references a particular record instance and creates an alias for that record.
If it returns success, then DBENTRY references the alias just created.
\verb|dbDeleteRecord| deletes either a single alias, or a single record instance and all the aliases that refer to it. \verb|dbDeleteAliases| finds and deletes all aliases that refer to the current record.
\verb|dbFreeRecords| deletes all record instances.

\subsection{Copy Record}

\begin{lstlisting}[language=C]
long dbCopyRecord(DBENTRY *pdbentry, char *newRecordName
    int overWriteOK)
\end{lstlisting}

\index{dbCopyRecord}
This routine copies the record instance currently referenced by \verb|DBENTRY| (it fails if DBENTRY references an alias).
Thus it creates a new record with the name \verb|newRecordName| that is of the same type as the original record and copies the original records field values to the new record.
If \verb|newRecordName| already exists and \verb|overWriteOK| is true, then the original \verb|newRecordName| is deleted and recreated.
If \verb|dbCopyRecord| completes successfully, DBENTRY references the new record.

\subsection{Rename Record}

\begin{lstlisting}[language=C]
long dbRenameRecord(DBENTRY *pdbentry, char *newname)
\end{lstlisting}

\index{dbRenameRecord}
This routine renames the record instance currently referenced by \verb|DBENTRY| (it fails if DBENTRY references an alias).
If \verb|dbRenameRecord| completes successfully, DBENTRY references the renamed record.

\subsection{Record Visibility}

These routines are intended for use by graphical configuration tools.

\begin{lstlisting}[language=C]
long dbVisibleRecord(DBENTRY *pdbentry);
long dbInvisibleRecord(DBENTRY *pdbentry);
int dbIsVisibleRecord(DBENTRY *pdbentry);
\end{lstlisting}

\index{dbVisibleRecord}
\index{dbInvisibleRecord}
\index{dbIsVisibleRecord}
Calling \verb|dbVisibleRecord| makes the record referenced by \verb|DBENTRY| visible.
\verb|dbInvisibleRecord| makes the record invisible.
\verb|dbIsVisibleRecord| returns TRUE if the record is visible, FALSE otherwise.

\subsection{Find Field}

\begin{lstlisting}[language=C]
long dbFindField(DBENTRY *pdbentry,char *pfieldName);
int dbFoundField(DBENTRY *pdbentry);

\end{lstlisting}

\index{dbFindField}
\index{dbFoundField}
Given that a record instance has been located, \verb|dbFindField| finds the specified field.
If it returns success, then \verb|DBENTRY| references that field.
\verb|dbFoundField| returns \verb|FALSE| if no field with the given name could be found, \verb|TRUE| if the field was located.

\subsection{Get/Put Field Values}

\begin{lstlisting}[language=C]
char *dbGetString(DBENTRY *pdbentry);
long dbPutString(DBENTRY *pdbentry,char *pstring);
char *dbVerify(DBENTRY *pdbentry,char *pstring);
char *dbGetRange(DBENTRY *pdbentry);
int   dbIsDefaultValue(DBENTRY *pdbentry);
\end{lstlisting}

\index{dbGetString}
\index{dbPutString}
\index{dbVerify}
\index{dbGetRange}
\index{dbIsDefaultValue}
These routines are used to get or change field values.
They work on any database field type except \verb|DCT_NOACCESS|.
\verb|dbVerify| returns \verb|NULL| if the string contains a valid value for this field or an error message if not.
Note that the strings returned are owned by the DBENTRY, so the next call passing that DBENTRY object that returns a string will overwrite the value returned by a previous call.
It is the caller's responsibility to copy the strings if the value must be kept.

\verb|DCT_MENU|, \verb|DCT_MENUFORM| and \verb|DCT_LINK_xxx| fields can be manipulated via routines described in the following sections.
If, however \verb|dbGetString| and \verb|dbPutString| are used, they do work correctly.
For these field types \verb|dbGetString| and \verb|dbPutString| are intended to be used only for creating and restoring versions of a database.

\section{Manipulating Menu Fields}

These routines should only be used for \verb|DCT_MENU| and \verb|DCT_MENUFORM| fields.
Thus they should only be called if \verb|dbFindField|, \verb|dbFirstField|, or \verb|dbNextField| has returned success and the field type is \verb|DCT_MENU| or \verb|DCT_MENUFORM|.

\subsection{Get Number of Menu Choices}

\begin{lstlisting}[language=C]
int  dbGetNMenuChoices(DBENTRY *pdbentry);
\end{lstlisting}

\index{dbGetNMenuChoices}
This routine returns the number of menu choices for menu.

\subsection{Get Menu Choice}

\begin{lstlisting}[language=C]
char **dbGetMenuChoices(DBENTRY *pdbentry);
\end{lstlisting}

\index{dbGetMenuChoices}
This routine returns the address of an array of pointers to strings which contain the menu choices.

\subsection{Get/Put Menu}

\begin{lstlisting}[language=C]
int  dbGetMenuIndex(DBENTRY *pdbentry);
long dbPutMenuIndex(DBENTRY *pdbentry,int index);
char *dbGetMenuStringFromIndex(DBENTRY *pdbentry,int index);
int dbGetMenuIndexFromString(DBENTRY *pdbentry,

char *choice);
\end{lstlisting}

\index{dbGetMenuIndex}
\index{dbPutMenuIndex}
\index{dbGetMenuStringFromIndex}
\index{dbGetMenuIndexFromString}
NOTE: These routines do not work if the current field value contains a macro definition.

\verb|dbGetMenuIndex| returns the index of the menu choice for the current field, i.e. it specifies which choice to which the 
field is currently set.
\verb|dbPutMenuIndex| sets the field to the choice specified by the index.

\verb|dbGetMenuStringFromIndex| returns the string value for a menu index.
If the index value is invalid NULL is returned.
\verb|dbGetMenuIndexFromString| returns the index for the given string.
If the string is not a valid choice -1 is returned.

\index{dbGetMenuStringFromIndex}
\index{dbGetMenuIndexFromString}
\subsection{Locate Menu}

\begin{lstlisting}[language=C]
dbMenu *dbFindMenu(DBBASE *pdbbase,char *name);
\end{lstlisting}

\index{dbFindMenu}
\verb|dbFindMenu| is most useful for runtime use but is a static database access routine.
This routine just finds a menu with the given name.

\section{Manipulating Link Fields}

\subsection{Link Types}

Links are the most complicated types of fields.
A link can be a constant, reference a field in another record, or can refer to a hardware device.
Two additional complications arise for hardware links.
The first is that field \verb|DTYP|, which is a menu field, determines if the \verb|INP| or \verb|OUT| field is a device link.
The second is that the information that must be specified for a device link is bus dependent.
In order to shelter database configuration tools from these complications the following is done for static database access.

\begin{itemize}
\item Static database access will treat \verb|DTYP| as a \verb|DCT_MENUFORM| field.

\item The information for the link field related to the \verb|DCT_MENUFORM| can be entered via a set of form manipulation routines associated with the \verb|DCT_MENUFORM| field.
Thus the link information can be entered via the \verb|DTYP| field rather than the link field.

\item The Form routines described in the next section can also be used with any link field.

\end{itemize}

Each link is one of the following types:

\begin{itemize}
\item \index{DCT\_LINK\_CONSTANT}\verb|DCT_LINK_CONSTANT|: Constant value.

\item \index{DCT\_LINK\_PV}\verb|DCT_LINK_PV|: A process variable link.

\item \index{DCT\_LINK\_FORM}\verb|DCT_LINK_FORM|: A link that can only be processed via the form routines described in the next section.

\end{itemize}

Database configuration tools can change any link between being a constant and a process variable link.
Routines are provided to accomplish these tasks.

The routines \verb|dbGetString|, \verb|dbPutString|, and \verb|dbVerify| can be used for link fields but the form routines can be used to provide a friendlier user interface.

\subsection{All Link Fields}

\begin{lstlisting}[language=C]
int  dbGetNLinks(DBENTRY *pdbentry);
long dbGetLinkField(DBENTRY *pdbentry,int index)
int  dbGetLinkType(DBENTRY *pdbentry);
\end{lstlisting}

\index{dbGetNLinks}
\index{dbGetLinkField}
\index{dbGetLinkType}
These are routines for manipulating \verb|DCT_xxxLINK| fields. \verb|dbGetNLinks| and \verb|dbGetLinkField| are used to walk 
through all the link fields of a record. \verb|dbGetLinkType| returns one of the values: \verb|DCT_LINK_CONSTANT|, 
\verb|DCT_LINK_PV|, \verb|DCT_LINK_FORM|, or the value -1 if it is called for an illegal field.

\subsection{Constant and Process Variable Links}

\begin{lstlisting}[language=C]
long dbCvtLinkToConstant(DBENTRY *pdbentry);
long dbCvtLinkToPvlink(DBENTRY *pdbentry);
\end{lstlisting}

\index{dbCvtLinkToConstant}
\index{dbCvtLinkToPvlink}
These routines should be used for modifying \verb|DCT_LINK_CONSTANT| or \verb|DCT_LINK_PV| links. They should not be used 
for \verb|DCT_LINK_FORM| links, which should be processed via the associated \verb|DCT_MENUFORM| field described above.

\subsection{Get Related Field}

\begin{lstlisting}[language=C]
char *dbGetRelatedField(DBENTRY *pdbentry)
\end{lstlisting}

This routine returns the field name of the related field for a \verb|DCT_MENUFORM| field.
If it is called for any other type of field it returns NULL.

\section{Manipulating Information Items}

Information items are stored as a list attached to each individual record instance. All routines listed in this section require 
that the DBENTRY argument refer to a valid record instance.

\subsection{Locate Item}

\begin{lstlisting}[language=C]
long dbFirstInfo(DBENTRY *pdbentry);
long dbNextInfo(DBENTRY *pdbentry);
long dbFindInfo(DBENTRY *pdbentry,const char *name);
\end{lstlisting}

\index{dbFirstInfo}
\index{dbNextInfo}
\index{dbFindInfo}
There are two ways to locate info items, by scanning through the list using first/next, or by asking for the item by name.
These routines set \verb|pdbentry| to refer to the info item and return 0, or return an error code if no info item is found.

\subsection{Get Item Name}

\begin{lstlisting}[language=C]
const char * dbGetInfoName(DBENTRY *pdbentry);
\end{lstlisting}

\index{dbGetInfoName}
Returns the name of the info item referred to by \verb|pdbentry|, or a NULL pointer if no item is referred to.

\subsection{Get/Set Item String Value}

\begin{lstlisting}[language=C]
const char * dbGetInfoString(DBENTRY *pdbentry);
long dbPutInfoString(DBENTRY *pdbentry,const char *string);
\end{lstlisting}

\index{dbGetInfoString}
\index{dbPutInfoString}
These routines provide access to the currently selected item's string value.
When changing the string value using \verb|dbPutInfoSting|, the character string provided will be copied, with additional memory being allocated as necessary.
Developers are advised not to make continuously repeated calls to \verb|dbPutInfoString| at IOC runtime as this could fragment 
the free memory heap.
The Put routine returns 0 if Ok or an error code; the Get routine returns NULL on error.

\subsection{Get/Set Item Pointer Value}

\begin{lstlisting}[language=C]
void * dbGetInfoPointer(DBENTRY *pdbentry);
long dbPutInfoPointer(DBENTRY *pdbentry, void *pointer);
\end{lstlisting}

\index{dbGetInfoPointer}
\index{dbPutInfoPointer}
Each info item includes space to store a single \verb|void*| pointer as well as the value string.
Applications using the info item may set this as often as they wish.
The Put routine returns 0 if Ok or an error code; the Get routine returns NULL on error.

\subsection{Create/Delete Item}

\begin{lstlisting}[language=C]
long dbPutInfo(DBENTRY *pdbentry,const char *name,const char *string);
long dbDeleteInfo(DBENTRY *pdbentry);
\end{lstlisting}

\index{dbPutInfo}
\index{dbDeleteInfo}
A new info item can be created by calling \verb|dbPutInfo|.
If an item by that name already exists its value will be replaced with the new string, otherwise storage is allocated and the name and value strings copied into it.
The function returns 0 on success, or an error code.

When calling \verb|dbDeleteInfo|, pdbentry must refer to the item to be removed (using \verb|dbFirstInfo|, \verb|dbNextInfo| or \verb|dbFindInfo|).
The function returns 0 on success, or an error code.

\subsection{Convenience Routine}

\begin{lstlisting}[language=C]
const char * dbGetInfo(DBENTRY *pdbentry,const char *name);
\end{lstlisting}

\index{dbGetInfo}
It is common to want to look up the value of a named info item in one call, and \verb|dbGetInfo| is provided for this purpose. It returns a NULL pointer if no info item exists with the given name.

\section{Find Breakpoint Table}

\begin{lstlisting}[language=C]
brkTable *dbFindBrkTable(DBBASE *pdbbase,char *name)
\end{lstlisting}

\index{dbFindBrkTable}
This routine returns the address of the specified breakpoint table.
It is normally used by the runtime breakpoint conversion routines so will not be discussed further.

\section{Dump Routines}

\begin{lstlisting}[language=C]
void dbDumpPath(DBBASE *pdbbase)
void dbDumpRecord(DBBASE *pdbbase,char *precordTypeName,int level);
void dbDumpMenu(DBBASE *pdbbase,char *menuName);
void dbDumpRecordType(DBBASE *pdbbase,char *recordTypeName);
void dbDumpField(DBBASE *pdbbase,char *recordTypeName,char *fname);
void dbDumpDevice(DBBASE *pdbbase,char *recordTypeName);
void dbDumpDriver(DBBASE *pdbbase);
void dbDumpRegistrar(DBBASE *pdbbase);
void dbDumpFunction(DBBASE *pdbbase);
void dbDumpVariable(DBBASE *pdbbase);
void dbDumpBreaktable(DBBASE *pdbbase,char *name);
void dbPvdDump(DBBASE *pdbbase,int verbose);
void dbReportDeviceConfig(DBBASE *pdbbase,FILE *report);
\end{lstlisting}

\index{dbDumpPath}
\index{dbDumpRecord}
\index{dbDumpMenu}
\index{dbDumpRecordType}
\index{dbDumpFldDes}
\index{dbDumpDevice}
\index{dbDumpDriver}
\index{dbDumpRegistrar}
\index{dbDumpFunction}
\index{dbDumpVariable}
\index{dbDumpBreaktable}
\index{dbPvdDump}
\index{dbReportDeviceConfig}
These routines are used to dump information about the database.
The routines \verb|dbDumpRecord|, \verb|dbDumpMenu|, \verb|dbDumpDriver|, \verb|dbDumpRegistrar| and \verb|dbDumpVariable| just call their corresponding \verb|dbWriteXxxFP| routine, specifying stdout for the file to write to.
\verb|dbDumpRecordType|, \verb|dbDumpField|, and \verb|dbDumpDevice| give internal information useful on an ioc.
These commands can be executed via iocsh, specifying pdbbase as the first argument.

\section{Examples}

\subsection{Expand Include}

\index{dbExpand}
This example is like the \verb|dbExpand| utility, except that it doesn't allow path or macro substitution options.
It reads a set of database definition files and writes all definitions to stdout.
All include statements appearing in the input files are expanded.

\begin{lstlisting}[language=C]
/* dbExpand.c */
#include <stdlib.h>
#include <stddef.h>
#include <stdio.h>
#include <epicsPrint.h>
#include <dbStaticLib.h>

DBBASE *pdbbase = NULL;

int main(int argc, char **argv)
{
    long status;
    int i;
    int arg;

    if (argc < 2) {
        printf("usage: expandInclude file1.db file2.db...\n");
        exit 0;
    }

    for (i = 1; i < argc; i++) {
        status = dbReadDatabase(&pdbbase, argv[i], NULL, NULL);
        if (!status) continue;
        fprintf(stderr, "For input file %s", argv[i]);
        errMessage(status, "from dbReadDatabase");
    }
    dbWriteMenuFP(pdbbase,stdout,0);
    dbWriteRecordTypeFP(pdbbase,stdout,0);
    dbWriteDeviceFP(pdbbase.stdout);
    dbWriteDriverFP(pdbbase.stdout);
    dbWriteRecordFP(pdbbase,stdout,0,0);
    return(0);
}

\end{lstlisting}

\subsection{dbDumpRecords}

\index{dbDumpRecords}
NOTE: This example is similar but not identical to the actual \verb|dbDumpRecords| routine.

The following example demonstrates how to use the database access routines.
The code shows how to iterate through the record types and instances and display field values.

\begin{lstlisting}[language=C]
void dbDumpRecords(DBBASE *pdbbase)
{
    DBENTRY  *pdbentry;
    long  status;

    pdbentry = dbAllocEntry(pdbbase);
    status = dbFirstRecordType(pdbentry);
    if (status) {
        printf("No record types\n");
        return;
    }
    while (!status) {
        printf("Record type: %s\n", dbGetRecordTypeName(pdbentry));
        status = dbFirstRecord(pdbentry);
        if (status)
            printf("  No record instances\n");
        else while (!status) {
            if (dbIsAlias(pdbentry))
                printf("  Alias: %s\n", dbGetRecordName(pdbentry));
            else {
                printf("  Record: %s\n", dbGetRecordName(pdbentry));
                status = dbFirstField(pdbentry, TRUE);
                if (status)
                    printf("    No fields\n");
                else while(!status) {
                    printf("    %s: %s\n", dbGetFieldName(pdbentry),
                        dbGetString(pdbentry));
                    status = dbNextField(pdbentry, TRUE);
                }
            }
            status = dbNextRecord(pdbentry);
        }
        status = dbNextRecordType(pdbentry);
    }
    printf("End of record types\n");
    dbFreeEntry(pdbentry);
}
\end{lstlisting}
