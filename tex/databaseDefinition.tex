\chapter{Database Definition}
\index{Database Definition}

\section{Overview}

This chapter describes \index{database definitions}database definitions. The following definitions are described:

\begin{itemize}\item Menu

\item Record Type

\item Device

\item Driver

\item Registrar

\item Variable

\item Function

\item Breakpoint Table

\item Record Instance

\end{itemize}Record Instances are fundamentally different from the other definitions. A file containing record instances should never 
contain any of the other definitions and vice-versa. Thus the following convention is followed:

\begin{itemize}\item \index{Database Definition File}\textbf{Database Definition File} - A file that contains any type of definition except record instances.

\item \index{Record Instance File}\textbf{Record Instance File} - A file that contains only record instance definitions.

\end{itemize}This chapter also describes utility programs which operate on these definitions

Any combination of definitions can appear in a single file or in a set of files related to each other via include files.

\section{Summary of Database Syntax}

\index{Database Format - Summary}The following summarizes the Database Definition syntax:

\begin{verbatim}path "path"
addpath "path"
include "filename"
#comment
menu(name) {
    include "filename"
    choice(choice_name,"choice_value")
    ...
}

recordtype(record_type) {
    include "filename"
    field(field_name,field_type) {
        asl(asl_level)
        initial("init_value")
        promptgroup(gui_group)
        prompt("prompt_value")
        special(special_value)
        pp(pp_value)
        interest(interest_level)
        base(base_type)
        size(size_value)
        extra("extra_info")
        menu(name)
    }
    ...
}

device(record_type,link_type,dset_name,"choice_string")

driver(drvet_name)

registrar(function_name)

variable(variable_name)

breaktable(name) {
    raw_value eng_value
    ...
}
\end{verbatim}The Following defines a Record Instance

\begin{verbatim}record(record_type,record_name) {
include "filename"
field(field_name,"value")
    alias(alias_name)
    info(info_name,"value")
    ...
}
alias(record_name,alias_name)
\end{verbatim}\section{General Rules for Database Definition}

\subsection{Keywords}

\index{Keywords}The following are keywords, i.e. they may not be used as values unless they are enclosed in quotes:

\begin{verbatim}
path
addpath
include
menu
choice
recordtype
field
device
driver
registrar
function
variable
breaktable
record
grecord
info
alias
\end{verbatim}
\index{path}
\index{addpath}
\index{include}
\index{menu}
\index{choice}
\index{recordtype}
\index{field}
\index{device}
\index{driver}
\index{registrar}
\index{function}
\index{variable}
\index{breaktable}
\index{record}
\index{grecord}
\index{info}
\index{alias}\subsection{Unquoted Strings}

\index{Unquoted String}In the summary section, some values are shown as quoted strings and some unquoted. The actual rule is that any string 
consisting of only the following characters does not have to be quoted unless it contains one of the above keywords:

\begin{verbatim}a-z A-Z 0-9 _ - : . [ ] < > ;
\end{verbatim}These are also the legal characters for process variable names. Thus in many cases quotes are not needed.

\subsection{Quoted Strings}

\index{Quoted String}A quoted string can contain any ascii character except the quote character \verb|"|. The quote character itself can given by using \verb|\| as an escape. For example \verb|"\""| is a quoted string containing the single character \verb|"|.

\subsection{Macro Substitution}

\index{Macro Substitution}Macro substitutions are permitted inside quoted strings. Macro instances take the form:

\begin{verbatim}$(name)
\end{verbatim}or

\begin{verbatim}${name}
\end{verbatim}There is no distinction between the use of parentheses or braces for delimiters, although the two must match for a given 
macro instance. The macro name can be made up from other macros, for example:

\begin{verbatim}$(name_$(sel))
\end{verbatim}A macro instance can also provide a default value that is used when no macro with the given name is defined. The default 
value can be defined in terms of other macros if desired, but cannot contain any unescaped comma characters. The syntax 
for specifying a default value is as follows:

\begin{verbatim}$(name=default)
\end{verbatim}Finally macro instances can also contain definitions of other macros, which can (temporarily) override any existing values 
for those macros but are in scope only for the duration of the expansion of this macro instance. These definitions consist 
of \verb|name=value| sequences separated by commas, for example:

\begin{verbatim}$(abcd=$(a)$(b)$(c)$(d),a=A,b=B,c=C,d=D)
\end{verbatim}\subsection{Escape Sequences}

\index{Escape Sequence}The database routines translate standard C escape sequences inside database field value strings only. The standard C 
escape sequences supported are:

\begin{verbatim}\a \b \f \n \r \t \v \\ \? \' \" \ooo \xhh
\end{verbatim}\verb|\ooo| represents an octal number with 1, 2, or 3 digits. \verb|\xhh| represents a hexadecimal number with 1 or 2 digits.

\subsection{Define before referencing}

No item can be referenced until it is defined. For example a \verb|recordtype| menu field can not reference a menu unless 
that menu definition has already been defined. Another example is that a record instance can not appear until the 
associated record type has been defined.

\subsection{Multiple Definitions}

\index{Multiple Definitions}If a menu, recordtype, device, driver, or breakpoint table is defined more than once, then only the first instance is used. 
Record instance definitions however are (normally) cumulative, so multiple instances of the same record may be loaded 
and each time a field value is encountered it replaces the previous value.

\subsection{filename extension}

\index{filename extension conventions}By convention:

\begin{itemize}\item Record instances files have the extension ``\verb|.db|"

\item Database definition files have the extension ``\verb|.dbd|"

\end{itemize}\subsection{path addpath}

\index{path - Database Definitions}
\index{addpath - Database Definitions}The path follows the standard Unix convention, i.e. it is a list of directory names separated by colons (Unix) or 
semicolons (windows).

Format:

\begin{verbatim}path "dir:dir...:dir"
addpath "dir:dir...:dir
\end{verbatim}NOTE: On windows the separator is \verb|;| instead of \verb|:|

The \verb|path| command specifies the current path. The \verb|addpath| appends directory names to the current path. The path is 
used to locate the initial database file and included files. An empty \verb|dir| at the beginning, middle, or end of a non-empty 
path string means the current directory. For example:

\begin{verbatim} nnn::mmm    # Current directory is between nnn and mmm
 :nnn        # Current directory is first
 nnn:        # Current directory is last
\end{verbatim}Utilities which load database files (\verb|dbExpand|, \verb|dbLoadDatabase|, etc.) allow the user to specify an initial path. The 
\verb|path| and \verb|addpath| commands can be used to change or extend the initial path.

The initial path is determined as follows:

\begin{description}\item If an initial path is specified, it is used. Else:

\item If the environment variable \verb|EPICS_DB_INCLUDE_PATH| is defined, it is used. Else:

\item the default path is ``.", i.e. the current directory.

\end{description}The path is used unless the filename contains a / or \textbackslash{}. The first directory containing the specified file is used.

\subsection{include}

\index{include - Database Definitions}Format:

\begin{verbatim}include "filename"
\end{verbatim}An include statement can appear at any place shown in the summary. It uses the path as specified above.

\subsection{comment}

\index{comment - Database Definitions}
\index{comment - Database Definitions}
\index{comment - Database Definitions}The comment symbol is ``\#". Whenever the comment symbol appears, it and all characters through the end of the line are 
ignored.

\section{Menu}

\index{menu - Database Definitions}Format:

\begin{verbatim}menu(name) {
    choice(choice_name,"choice_value")
    ...
}
\end{verbatim}Where:

\begin{description}\item name - Name for menu. This is the unique name identifying the menu. If duplicate definitions are specified, only 
the first is used.

\item choice\_name - The name placed in the \verb|enum| generated by \verb|dbToMenuH| or \verb|dbToRecordtypeH|

\item choice\_value - The value associated with the choice.

\end{description}Example:

\begin{verbatim}
menu(menuYesNo) {
    choice(menuYesNoNO,"NO")
    choice(menuYesNoYES,"YES")
}
\end{verbatim}\section{Record Type}

\index{record type - Database Definitions}\subsection{Format:}

\begin{verbatim}
recordtype(record_type) {
    field(field_name,field_type) {
    asl(asl_level)
    initial("init_value")
    promptgroup(gui_group)
    prompt("prompt_value")
    special(special_value)
    pp(pp_value)
    interest(interest_level)
    base(base_type)
    size(size_value)
    extra("extra_info")
    menu("name")
    }
    ...
}
\end{verbatim}\subsection{rules}

\index{field definition rules}\begin{itemize}\item \index{asl - field definition rules}asl - Access Security Level. The default is \verb|ASL1|. Access Security is discussed in a later chapter. Only two values 
are permitted for this field (\verb|ASL0| and \verb|ASL1|). Fields which operators normally change are assigned \verb|ASL0|. Other 
fields are assigned \verb|ASL1|. For example, the \verb|VAL| field of an analog output record is assigned \verb|ASL0| and all other 
fields \verb|ASL1|. This is because only the \verb|VAL| field should be modified during normal operations.

\item \index{initial - field definition rules}initial - Initial Value.

\item \index{promptgroup - field definition rules}promptgroup - Prompt group to which field belongs. This is for use by Database Configuration Tools. This is 
defined only for fields that can be given values by database configuration tools. File \verb|guigroup|.\verb|h| contains all 
possible definitions. The different groups allow database configuration tools to present the user with groups of 
fields rather than all prompt fields. I don't know of any tool that currently uses groups.

\item \index{prompt - field definition rules}prompt - A prompt string for database configuration tools. Optional if \verb|promptgroup| is not defined.

\item \index{special - field definition rules}special - If specified, then special processing is required for this field at run time.

\item \index{pp - field definition rules}pp - Should a passive record be processed when Channel Access writes to this field? The default is \verb|NO|.

\item \index{interest - field definition rules}interest - Only used by the \verb|dbpr| shell command.

\item \index{base - field definition rules}base - For integer fields, a base of \verb|DECIMAL| or \verb|HEX| can be specified. The default is \verb|DECIMAL|.

\item \index{size - field definition rules}size - Must be specified for \verb|DBF_STRING| fields.

\item \index{extra - field definition rules}extra - Must be specified for \verb|DBF_NOACCESS| fields.

\item \index{menu - field definition rules}menu - Must be specified for \verb|DBF_MENU| fields. It is the name of the associated menu.

\end{itemize}\subsection{definitions}

\begin{itemize}\item \index{record\_type - record type definition}record\_type - The unique name of the record type. If duplicates are specified, only the first definition is used.

\item \index{field\_name - field definition}field\_name - The field name, which must be a valid C identifier so must start with a letter or underscore and 
thereafter may also contain digits. When include files are generated, the field name is converted to lower case. 
Previous versions of EPICS required that field name be a maximum of four characters. Although this restriction  no 
longer exists, problems may arise with some Channel Access clients if longer field names are chosen.

\item \index{field\_type - field definition}field\_type - This must be one of the following values:

\begin{itemize}
\item \verb|DBF_STRING|
\item \verb|DBF_CHAR|
\item \verb|DBF_UCHAR|
\item \verb|DBF_SHORT|
\item \verb|DBF_USHORT|
\item \verb|DBF_LONG|
\item \verb|DBF_ULONG|
\item \verb|DBF_FLOAT|
\item \verb|DBF_DOUBLE|
\item \verb|DBF_ENUM|
\item \verb|DBF_MENU|
\item \verb|DBF_DEVICE|
\item \verb|DBF_INLINK|
\item \verb|DBF_OUTLINK|
\item \verb|DBF_FWDLINK|
\item \verb|DBF_NOACCESS|
\end{itemize}
\item \index{asl\_level - field definition}asl\_level - This must be one of the following values:
\begin{itemize}
\item \verb|ASL0|
\item \verb|ASL1|  (default value)
\end{itemize}
\item \index{init\_value - field definition}init\_value - A legal value for data type.
\item \index{prompt\_value - field definition}prompt\_value - A prompt value for database configuration tools.
\item \index{gui\_group - field definition}gui\_group - This must be one of the following:
\begin{itemize}
\item \verb|GUI_COMMON|
\item \verb|GUI_ALARMS|
\item \verb|GUI_BITS1|
\item \verb|GUI_BITS2|
\item \verb|GUI_CALC|
\item \verb|GUI_CLOCK|
\item \verb|GUI_COMPRESS|
\item \verb|GUI_CONVERT|
\item \verb|GUI_DISPLAY|
\item \verb|GUI_HIST|
\item \verb|GUI_INPUTS|
\item \verb|GUI_LINKS|
\item \verb|GUI_MBB|
\item \verb|GUI_MOTOR|
\item \verb|GUI_OUTPUT|
\item \verb|GUI_PID|
\item \verb|GUI_PULSE|
\item \verb|GUI_SELECT|
\item \verb|GUI_SEQ1|
\item \verb|GUI_SEQ2|
\item \verb|GUI_SEQ3|
\item \verb|GUI_SUB|
\item \verb|GUI_TIMER|
\item \verb|GUI_WAVE|
\item \verb|GUI_SCAN|

NOTE: \verb|GUI| types were invented with the intention of allowing database configuration tools to prompt for 
groups of fields and when a user selects a group the fields within the group. Since this feature has seldom 
been used, many record types have not assigned the correct GUI groups to some fields.

\end{itemize}\item \index{special\_value - field definition}special\_value must be one of the following:

\begin{itemize}
\item An integer value greater than 103. In this case, the record support special routine is called whenever the field 
is modified by database access. This feature is present only for compatibility. New support modules should 
use \verb|SPC_MOD|.

The following value disallows access to field.

\item \verb|SPC_NOMOD| - This means that field can not be modified at runtime except by the record/device support 
modules for the record type.

\index{SPC\_NOMOD}

The following values are used for database common. They must NOT be used for record specific fields.

\verb|SPC_SCAN| - Scan related field.

\item \index{SPC\_SCAN}\verb|SPC_ALARMACK| - Alarm acknowledgment field.

\item \index{SPC\_ALARMACK}\verb|SPC_AS| - Access security field.

\index{SPC\_AS}

The following value is used if record support wants to trap \verb|dbNameToAddr| calls.

\item \verb|SPC_DBADDR| - This is set if the record support \verb|cvt_dbaddr| routine should be called whenever 
\verb|dbNameToAddr| is called, i.e. when code outside record/device support want to access the field.

\index{SPC\_DBADDR}

The following values all result in the record support special routine being called whenever database access 
modifies the field. The only reason for multiple values is that originally it seemed like a good idea. New 
support modules should only use \verb|SPC_MOD|.

\item \verb|SPC_MOD| - Notify when modified, i.e. call the record support special routine whenever the field is modified 
by database access.

\item \index{SPC\_MOD}\verb|SPC_RESET| - a reset field is being modified.

\item \index{SPC\_RESET}\verb|SPC_LINCONV| - A linear conversion field is being modified.

\item \index{SPC\_LINCONV}\verb|SPC_CALC| - A calc field is being modified.

\index{SPC\_CALC}
\end{itemize}

\item \index{pp\_value - field definition}pp\_value - Should a passive record be processed when Channel Access writes to this field?  The allowed values 
are:

\begin{itemize}
\item \verb|NO| (default)
\item \verb|YES|
\end{itemize}

\item \index{interest\_level - field definition}interest\_level - An interest level for the \verb|dbpr| command.

\item \index{base - field definition}base - For integer type fields, the default base. The legal values are:

\begin{itemize}
\item \verb|DECIMAL| (Default)
\item \verb|HEX|
\end{itemize}

\item \index{size\_value - field definition}size\_value - The number of characters for a \verb|DBF_STRING| field.

\item \index{extra\_info - field definition}extra\_info - For \verb|DBF_NOACCESS| fields, this is the C language definition for the field. The definition must end 
with the fieldname in lower case.

\end{itemize}

\subsection{Example}

The following is the definition of the binary input record.

\begin{verbatim}
recordtype(bi) {
    include "dbCommon.dbd" 
        field(INP,DBF_INLINK) {
        prompt("Input Specification")
        promptgroup(GUI_INPUTS)
        interest(1)
    }
    field(VAL,DBF_ENUM) {
        prompt("Current Value")
        promptgroup(GUI_INPUTS)
        asl(ASL0)
        pp(TRUE)
    }
    field(ZSV,DBF_MENU) {
        prompt("Zero Error Severity")
        promptgroup(GUI_ALARMS)
        pp(TRUE)
        interest(1)
        menu(menuAlarmSevr)
    }
    field(OSV,DBF_MENU) {
        prompt("One Error Severity")
        promptgroup(GUI_BITS1)
        pp(TRUE)
        interest(1)
        menu(menuAlarmSevr)
    }
    field(COSV,DBF_MENU) {
        prompt("Change of State Svr")
        promptgroup(GUI_BITS2)
        pp(TRUE)
        interest(1)
        menu(menuAlarmSevr)
    }
    field(ZNAM,DBF_STRING) {
        prompt("Zero Name")
        promptgroup(GUI_CALC)
        pp(TRUE)
        interest(1)
        size(20)
    }
    field(ONAM,DBF_STRING) {
        prompt("One Name")
        promptgroup(GUI_CLOCK)
        pp(TRUE)
        interest(1)
        size(20)
    }
    field(RVAL,DBF_ULONG) {
        prompt("Raw Value")
        pp(TRUE)
    }
    field(ORAW,DBF_ULONG) {
        prompt("prev Raw Value")
        special(SPC_NOMOD)
        interest(3)
    }
    field(MASK,DBF_ULONG) {
        prompt("Hardware Mask")
        special(SPC_NOMOD) 
        interest(1)
    }
    field(LALM,DBF_USHORT) {
        prompt("Last Value Alarmed")
        special(SPC_NOMOD)
        interest(3)
    }
    field(MLST,DBF_USHORT) {
        prompt("Last Value Monitored")
        special(SPC_NOMOD)
        interest(3)
    }
    field(SIOL,DBF_INLINK) {
        prompt("Sim Input Specifctn")
        promptgroup(GUI_INPUTS)
        interest(1)
    }
    field(SVAL,DBF_USHORT) {
        prompt("Simulation Value")
    }
    field(SIML,DBF_INLINK) {
        prompt("Sim Mode Location")
        promptgroup(GUI_INPUTS)
        interest(1)
    }
    field(SIMM,DBF_MENU) {
        prompt("Simulation Mode")
        interest(1)
        menu(menuYesNo)
    }
    field(SIMS,DBF_MENU) {
        prompt("Sim mode Alarm Svrty")
        promptgroup(GUI_INPUTS)
        interest(2)
        menu(menuAlarmSevr)
    }
}
\end{verbatim}

\section{Device}

\index{device - Database Definitions}\subsection{Format:}

\begin{verbatim}device(record_type,link_type,dset_name,"choice_string")
\end{verbatim}\subsection{definitions}

\begin{itemize}\item \index{record\_type - device definition}record\_type - Record type. The combination of \verb|record_type| and  \verb|choice_string| must be unique. If the 
same combination appears multiple times, the first definition is used. 

\item \index{link\_type - device definition}link\_type - Link type. This must be one of the following:

\begin{itemize}

\item \verb|CONSTANT|

\item \verb|PV_LINK|

\item \verb|VME_IO|

\item \verb|CAMAC_IO|

\item \verb|AB_IO|

\item \verb|GPIB_IO|

\item \verb|BITBUS_IO|

\item \verb|INST_IO|

\item \verb|BBGPIB_IO|

\item \verb|RF_IO|

\item \verb|VXI_IO|

\end{itemize}

\item \index{dset\_name - device definition}dset\_name -  The exact name of the device support entry table without the trailing ``\verb|DSET|". Duplicates are not 
allowed.

\item \index{choice\_string - device definition}choice\_string  Choice string for database configuration tools. Note that it must be enclosed in ``". Note that for a 
given record type, each \verb|choice_string| must be unique.

\end{itemize}\subsection{Examples}

\begin{verbatim}device(ai,CONSTANT,devAiSoft,"Soft Channel")
device(ai,VME_IO,devAiXy566Se,"XYCOM-566 SE Scanned")
\end{verbatim}\section{Driver}

\index{driver - database definition}\subsection{Format:}

\begin{verbatim}driver(drvet_name)
\end{verbatim}\subsection{Definitions}

\begin{itemize}\item \index{drvet\_name - driver definition}drvet\_name - If duplicates are defined, only the first is used.

\end{itemize}\subsection{Examples}

\begin{verbatim}driver(drvVxi)
driver(drvXy210)
\end{verbatim}\section{Registrar Declaration}

\index{registrar - Database Defintions}

\subsection{Format:}

\begin{verbatim}registrar(function_name)
\end{verbatim}\subsection{Definitions}

\begin{itemize}\item \index{function\_name - registrar definition}function\_name - The name of an C function that accepts no arguments, returns void and has been exported from 
its source file with an \index{epicsExportRegistrar}epicsExportRegistrar declaration, e.g.

\end{itemize}\begin{verbatim}
static void myRegistrar(void);
epicsExportRegistrar(myRegistrar);
\end{verbatim}This can be used to register functions for use by subroutine records or that can be invoked from iocsh. The example 
application described in Chapter 2.2, ``Example IOC Application" on page11" provides an example of how to register 
functions for subroutine records.

\subsection{Example}

\begin{verbatim}registrar(asSub)
\end{verbatim}\section{Variable Declaration}

\index{variable - Database Definitions}
\index{Variable Declaration}\subsection{Format:}

\begin{verbatim}variable(variable_name[, type])
\end{verbatim}\subsection{Definitions}

\begin{itemize}\item \index{variable\_name - variable definition}variable\_name - The name of a C variable which has been exported from its source file with an 
\index{epicsExportAddress}epicsExportAddress declaration.

\item \index{type - variable definition}type - The C variable type.  If not present, int is assumed.  Currently only int and double  variables are supported.

\end{itemize}This registers a diagnostic/configuration variable for device and driver support or a subroutine record subroutine so that 
the variable can be read and set with the iocsh var command (Section on page251). The example application described in 
Section2.2 on page11 provides an example of how to register a debug variable for a subroutine record.

\subsection{Example}

In an application C source file:

\begin{verbatim}#include <epicsExport.h>

static double myParameter;
epicsExportAddress(double, myParameter);
\end{verbatim}In an application database definition file:

\begin{verbatim}variable(myParameter, double)
\end{verbatim}\section{Function Declaration}

\index{function - Database Definitions}\subsection{Format:}

\begin{verbatim}function(function_name)
\end{verbatim}\subsection{Definitions}

\begin{itemize}\item \index{function\_name - function definition}function\_name - The name of a C function which has been exported from its source file with an 
\index{epicsRegisterFunction}epicsRegisterFunction declaration.

\end{itemize}This registers a function so that it can be found in the function registry for use by record types such as sub or aSub which 
refer to the function by name. The example application described in Chapter 2.2, ``Example IOC Application" on page11 
provides an example of how to register functions for a subroutine record.

\subsection{Example}

In an application C source file:

\begin{verbatim}    #include <epicsExport.h>
    #include <registryFunction.h>
    
    static long myFunction(void *argp) {
        /* my code ... */
    }
    epicsRegisterFunction(myFunction);
\end{verbatim}In an application database definition file:

\begin{verbatim}    function(myFunction)
\end{verbatim}\section{Breakpoint Table}

\index{breakpoint table - Database Definitions}\subsection{Format:}

\begin{verbatim}breaktable(name) {
raw_value eng_value
...
}
\end{verbatim}\subsection{Definitions}

\begin{itemize}\item \index{name - breakpoint table}name - Name, which must be alpha-numeric, of the breakpoint table. If duplicates are specified the first is used.

\item \index{raw\_value - breakpoint table}raw\_value - The raw value, i.e. the actual ADC value associated with the beginning of the interval.

\item \index{eng\_value - breakpoint table}eng\_value - The engineering value associated with the beginning of the interval.

\end{itemize}\subsection{Example}

\begin{verbatim}breaktable(typeJdegC) {
    0.000000 0.000000
    365.023224 67.000000
    1000.046448 178.000000
    3007.255859 524.000000
    3543.383789 613.000000
    4042.988281 692.000000
    4101.488281 701.000000
}
\end{verbatim}\section{Record Instance}

\index{record instance - Database Definitions}\subsection{Format:}

\begin{verbatim}
record(record_type, record_name) {
    alias(alias_name)
    field(field_name, "field_value")
    info(info_name, "info_value")
    ...
}
alias(record_name, alias_name)
\end{verbatim}
\subsection{definitions}
\begin{itemize}\item \index{record\_type - record instance definition}record\_type - The record type. 

\item \index{record\_name - record instance definition}record\_name - The record name. This must be composed of the following characters:\\
\verb|a-z A-Z 0-9 _ - + : [ ] < > ;|\\
NOTE: If macro substitutions are used the name must be quoted.

If duplicate definitions are given for the same record, then the last value given for each field is the value assigned 
to the field. 

\item \index{alias\_name - record instance definition}alias\_name - an alternate name for the record, following the same rules as record\_name.

\item \index{field\_name - record instance definition}field\_name - A field name

\item \index{field value - record instance definition}field\_value - Depends on field type. Inside the double quotes the field value string may contain escaped C89 
characters such as \textbackslash{}", \textbackslash{}t, \textbackslash{}n, \textbackslash{}064 and \textbackslash{}x7e, and these will be translated appropriately when loading the database.

\begin{itemize}

\item \verb|DBF_STRING|

Any ASCII string. If it exceeds the field length, it will be truncated.

\item \verb|DBF_CHAR|, \verb|DBF_UCHAR|, \verb|DBF_SHORT|, \verb|DBF_USHORT|, \verb|DBF_LONG|, \verb|DBF_ULONG|

A string that represents a valid integer. The standard C conventions are applied, i.e. a leading 0 means the 
value is given in octal and a leading 0x means that value is given in hex.

\item \verb|DBF_FLOAT|, \verb|DBF_DOUBLE|

The string must represent a valid floating point number.

\item \verb|DBF_MENU|

The string must be one of the valid choices for the associated menu.

\item \verb|DBF_DEVICE|

The string must be one of the valid device choice strings.

\item \verb|DBF_INLINK|, \verb|DBF_OUTLINK|

\verb|NOTES:|

\begin{itemize}
\item In the field is INP or OUT then it is associated with field DTYP. Other DBF\_INLINK and 
DBF\_OUTLINK fields can be either CONSTANT or PV\_LINKs

\item \verb|DTYP must be set before its associated INP or OUT field.|

\item Choosing the DTYP implicitly chooses a bus type.

\item A DTYP of CONSTANT can be either a constant or a PV\_LINK.

\end{itemize}

The allowed value depends on the bus type as follows:

\begin{itemize}

\item \verb|CONSTANT|

A constant valid for the field associated 

\item \index{CONSTANT - link field value}\verb|PV_LINK|

A value of the form:



    record.field process maximize

\index{PV\_LINK - link field value}

\verb|field|, \verb|process|, and \verb|maximize| are optional. 

The default value for \verb|field| is \verb|VAL|.

\verb|process| can have one of the following values:

\begin{itemize}

\item \verb|NPP| - No Process Passive (Default)

\item \verb|PP| - Process Passive

\item \verb|CA| - Force link to be a channel access link

\item \verb|CP| - CA and process on monitor

\item \verb|CPP| - CA and process on monitor if record is passive

NOTES:

CP and CPP are valid only for INLINKs.

FWD\_LINKs can be PP or CA. If a FWD\_LINK is a channel access link it must reference the 
PROC field.

\end{itemize}

\verb|maximize| can have one of the following values

\begin{itemize}

\item \verb|NMS| - No Maximize Severity (Default)

\item \verb|MS| - Maximize severity

\end{itemize}

\item \verb|VME_IO|

\index{VME\_IO - link field value}\#Ccard Ssignal @parm

where:

\verb|card| - the card number of associated hardware module.
\verb|signal| - signal on card
\verb|parm| - An arbitrary character string of up to 31 characters. This field is optional and is device specific.

\item \verb|CAMAC_IO|

\index{CAMAC\_IO - link field value}\#Bbranch Ccrate Nstation Asubaddress Ffunction @parm

\verb|branch|, \verb|crate|, \verb|station|, \verb|subaddress|, and \verb|function| should be obvious to \verb|camac| users. 
\verb|Subaddress| and \verb|function| are optional (0 if not given). \verb|Parm| is also optional and is device 
dependent (25 characters max).

\item \verb|AB_IO|

\index{AB\_IO - link field value}\#Llink Aadapter Ccard Ssignal @parm

\verb|link| - Scanner., i.e. vme scanner number

\verb|adapter| - Adapter. Allen Bradley also calls this rack

\verb|card| - Card within Allen Bradley Chassis

\verb|signal| - signal on card

\verb|parm| - An optional character string that is device dependent(27 char max)

\item \verb|GPIB_IO|

\index{GPIB\_IO - link field value}\#Llink Aaddr @parm

\verb|link| - gpib link, i.e. interface

\verb|addr| - GPIB address

\verb|parm| - device dependent character string (31 char max)

\item \verb|BITBUS_IO|

\index{BITBUS\_IO - link field value}\#Llink Nnode Pport Ssignal @parm

\verb|link| - link, i.e.  vme bitbus interface.

\verb|node| - bitbus node

\verb|port| - port on the node

\verb|signal| - signal on port

\verb|parm| - device specific character string(31 char max)

\item \verb|INST_IO|

\index{INST\_IO - link field value}@parm

\verb|parm| - Device dependent character string

\item \verb|BBGPIB_IO|

\index{BBGPIB\_IO - link field value}\#Llink Bbbaddr Ggpibaddr @parm

\verb|link| - link, i.e. vme bitbus interface.

\verb|bbadddr| - bitbus address

\verb|gpibaddr| - gpib address

\verb|parm| - optional device dependent character string(31 char max)

\item \verb|RF_IO|

\index{RF\_IO - link field value}\verb|#Rcryo Mmicro Ddataset Eelemen|t

\item \verb|VXI_IO|

\index{VXI\_IO - link field value}\#Vframe Cslot Ssignal @parm (Dynamic addressing)

     or

\verb|#Vla Signal @parm  (Static Addressing|)

\verb|frame| - VXI frame number

\verb|slot| - Slot within VXI frame

\verb|la| - Logical Address

\verb|signal| - Signal Number

\verb|parm| - device specific character string(25 char max)

\end{itemize}

\item \verb|DBF_FWDLINK|

This is either not defined or else is a \verb|PV_LINK|. See above for definitions.

\end{itemize}

\item \index{info\_name - record instance definition}info\_name - The name of an information item related to this record. See section 6.13 below for more on 
information items.

\item \index{info\_value - record instance definition}info\_value - Any ASCII string. IOC applications using this information item may place additional restrictions on 
the contents of the string.

\end{itemize}

\subsection{Examples}

\begin{verbatim}
record(ai,STS_AbAiMaS0) {
    field(SCAN,".1 second")
    field(DTYP,"AB-1771IFE-4to20MA")
    field(INP,"#L0 A2 C0 S0 F0 @")
    field(PREC,"4")
    field(LINR,"LINEAR")
    field(EGUF,"20")
    field(EGUL,"4")
    field(EGU,"MilliAmps")
    field(HOPR,"20")
    field(LOPR,"4")
}
record(ao,STS_AbAoMaC1S0) {
    field(DTYP,"AB-1771OFE")
    field(OUT,"#L0 A2 C1 S0 F0 @")
    field(LINR,"LINEAR")
    field(EGUF,"20")
    field(EGUL,"4")
    field(EGU,"MilliAmp")
    field(DRVH,"20")
    field(DRVL,"4")
    field(HOPR,"20")
    field(LOPR,"4")
    info(autosaveFields,"VAL")
}
record(bi,STS_AbDiA0C0S0) {
    field(SCAN,"I/O Intr")
    field(DTYP,"AB-Binary Input")
    field(INP,"#L0 A0 C0 S0 F0 @")
    field(ZNAM,"Off")
    field(ONAM,"On")
}
\end{verbatim}

\section{Record Information Item}

\index{Information item}Information items provide a way to attach named string values to individual record instances that are loaded at the same 
time as the record definition. They can be attached to any record without having to modify the record type, and can be 
retrieved by programs running on the IOC (they are not visible via Channel Access at all). Each item attached to a single 
record must have a unique name by which it is addressed, and database access provides routines to allow a record's info 
items to be scanned, searched for, retrieved and set. At runtime a \verb|void*| \index{Infomation item pointer}pointer can also be associated with each item, 
although only the string value can be initialized from the record definition when the database is loaded.

\section{Record Attribute}

\index{record attribute}Each record type can have a set of record attributes. Each attribute is a ``\index{Psuedo field}psuedo" field that can be accessed via database 
and channel access. An attribute is given a name the acts like a field name which has the same value for every instance of 
the record type. Two attributes are generated automatically for each record type: RTYP and VERS. The value for RTYP is 
the record type name. The default value for VERS is ``none specified", which can be changed by record support. Record 
support can call the following routine to create new attributes or change existing attributes:

\begin{verbatim}long dbPutAttribute(char *recordTypename,

       char *name,char*value)
\end{verbatim}\index{dbPutAttribute}The arguments are:

\begin{description}\item \verb|recordTypename| - The name of recordtype.

\item \verb|name| - The attribute name, i.e. the psuedo field name.

\item \verb|value| - The value assigned to the attribute.

\end{description}\section{Breakpoint Tables - Discussion}

\index{Breakpoint Tables}The menu \verb|menuConvert| is used for field \verb|LINR| of the \verb|ai| and \verb|ao| records. These records allow raw data to be converted 
to/from engineering units via one of the following:

\index{menuConvert}
\index{LINR}\begin{enumerate}\item No Conversion.

\item \index{Slope Conversion}Slope Conversion.

\item \index{Linear Conversion}Linear Conversion.

\item Breakpoint table.

\end{enumerate}Other record types can also use this feature. The first choice specifies no conversion; the second and third are both linear 
conversions, the difference being that for Slope conversion the user specifies the conversion slope and offset directly, 
whereas for Linear conversions these are calculated by the device support from the requsted Engineering Units range and 
the device support's knowledge of the hardware conversion range. The remaining choices are assumed to be the names of 
breakpoint tables. If a breakpoint table is chosen, the record support modules calls \verb|cvtRawToEngBpt| or 
\verb|cvtEngToRawBpt|. You can look at the \verb|ai| and \verb|ao| record support modules for details.

If a user wants to add additional breakpoint tables, then the following should be done:

\begin{itemize}\item Copy the \verb|menuConvert|.\verb|dbd| file from EPICS \verb|base|/\verb|src/bpt|

\item Add definitions for new breakpoint tables to the end

\item Make sure modified \verb|menuConvert|.\verb|dbd| is loaded  into the IOC instead of EPICS version.

\end{itemize}It is only necessary to load a breakpoint file if a record instance actually chooses it. It should also be mentioned that the 
Allen Bradley IXE device support misuses the \verb|LINR| field. If you use this module, it is very important that you do not 
change any of the EPICS supplied definitions in \verb|menuConvert|.\verb|dbd|. Just add your definitions at the end.

If a breakpoint table is chosen, then the corresponding breakpoint file must be loaded into the IOC before \verb|iocInit| is 
called.

Normally, it is desirable to directly create the breakpoint tables. However, sometimes it is desirable to create a breakpoint 
table from a table of raw values representing equally spaced engineering units. A good example is the Thermocouple 
tables in the OMEGA Engineering, INC Temperature Measurement Handbook. A tool \verb|makeBpt| is provided to convert 
such data to a breakpoint table.

The format for generating a breakpoint table from a data table of raw values corresponding to equally spaced engineering 
values is:

\begin{verbatim}!comment line
<header line>
<data table>
\end{verbatim}The header line contains the following information:

\begin{itemize}\item Name: An alphanumeric ascii string specifying the breakpoint table name

\item Low Value Eng: Engineering Units Value for first breakpoint table entry

\item Low Value Raw: Raw value for first breakpoint table entry

\item High Value Eng: Engineering Units: Highest Value desired

\item High Value Raw: Raw Value for High Value Eng

\item Error: Allowed error (Engineering Units)

\item First Table: Engineering units corresponding to first data table entry

\item Last Table: Engineering units corresponding to last data table entry

\item Delta Table: Change in engineering units per data table entry

\end{itemize} An example definition is:

\begin{verbatim}"TypeKdegF" 32 0 1832 4095 1.0 -454 2500 1
<data table>
\end{verbatim}The breakpoint table can be generated by executing

\begin{verbatim}makeBpt bptXXX.data
\end{verbatim}The input file must have the extension of data. The output filename is the same as the input filename with the extension of 
dbd.

Another way to create the breakpoint table is to include the following definition in a Makefile.Vx:

\begin{verbatim}BPTS += bptXXX.dbd
\end{verbatim}NOTE: This requires the naming convention that all data tables are of the form bpt\textless{}name\textgreater{}.data and a breakpoint table 
bpt\textless{}name\textgreater{}.dbd.

\section{Menu and Record Type Include File Generation.}

\index{Include File Generation}\subsection{Introduction}

Given a file containing menus, \verb|dbToMenuH| generates an include file that can be used by any code which uses the 
associated menus. Given a file containing any combination of menu definitions and record type definitions, 
\verb|dbToRecordtypeH| generates an include file that can be used by any code which uses the menus and record type.

\index{dbToMenuH}
\index{dbToRecordtypeH}EPICS base uses the following conventions for managing menu and recordtype definitions. Users generating local record 
types are encouraged to do likewise.

\begin{itemize}\item Each menu that is either for fields in database common (for example \verb|menuScan|) or is of global use (for example 
\verb|menuYesNo|) is defined in a separate file. The name of the file is the same as the menu name with an extension of 
\verb|dbd|. The name of the generated include file is the menu name with an extension of \verb|h|. Thus \verb|menuScan| is defined 
in a file \verb|menuScan|.\verb|dbd| and the generated include file is named \verb|menuScan|.\verb|h|

\item Each record type definition is defined in a separate file. In addition, this file contains any menu definitions that are 
used only by that record type. The name of the file is the same as the recordtype name followed by \verb|Record|.\verb|dbd|. 
The name of the generated include file is the same name with an extension of \verb|h|. Thus \verb|aoRecord| is defined in a 
file \verb|aoRecord|.\verb|dbd| and the generated include file is named \verb|aoRecord|.\verb|h|. Since \verb|aoRecord| has a private menu 
called \verb|aoOIF|, the \verb|dbd| file and the generated include file have definitions for this menu. Thus for each record type, 
there are two source files (\verb|xxxRecord|.\verb|dbd| and \verb|xxxRecord|.\verb|c|) and one generated file (\verb|xxxRecord|.\verb|h|). 

\end{itemize}Before continuing, it should be mentioned that Application Developers don't have to execute \verb|dbToMenuH| or 
\verb|dbToRecordtypeH|. If a developer uses the proper naming conventions, it is only necessary to add definitions to their 
\verb|Makefile|.\verb| |Consult the chapter on the EPICS Build Facility for details..

\subsection{dbToMenuH}

\index{dbToMenuH}This tool is executed as follows:

\begin{verbatim}dbToMenuH -Idir -Smacsub menuXXX.dbd
\end{verbatim}It generates a file which has the same name as the input file but with an extension of \verb|h|. Multiple \verb|-I| options can be 
specified for an include path and multiple \verb|-S| options for macro substitution.

For example \verb|menuPriority|.\verb|dbd|, which contains the definitions for processing priority contains:

\begin{verbatim}menu(menuPriority) {
    choice(menuPriorityLOW,"LOW")
    choice(menuPriorityMEDIUM,"MEDIUM")
    choice(menuPriorityHIGH,"HIGH")
}
\end{verbatim}The include file, \verb|menuPriority|.\verb|h|, generated by \verb|dbToMenuH| contains:

\begin{verbatim}#ifndef INCmenuPriorityH
#define INCmenuPriorityH
typedef enum {
    menuPriorityLOW,
    menuPriorityMEDIUM,
    menuPriorityHIGH,
}menuPriority;
#endif /*INCmenuPriorityH*/

\end{verbatim}Any code that needs to use the priority menu values should use these definitions.

\subsection{dbToRecordtypeH}

\index{dbToRecordtypeH}This tool is executed as follows:

\begin{verbatim}dbTorecordtypeH -Idir -Smacsub xxxRecord.dbd
\end{verbatim}It generates a file which has the same name as the input file but with an extension of \verb|h|. Multiple \verb|-I| options can be 
specified for an include path and multiple \verb|-S| options for macro substitution.

For example \verb|aoRecord|.\verb|dbd|, which contains the definitions for the analog output record contains:

\begin{verbatim}
menu(aoOIF) {
    choice(aoOIF_Full,"Full")
    choice(aoOIF_Incremental,"Incremental")
}
recordtype(ao) {
    include "dbCommon.dbd"
    field(VAL,DBF_DOUBLE) {
    prompt("Desired Output")
        asl(ASL0)
        pp(TRUE)
    }
    field(OVAL,DBF_DOUBLE) {
        prompt("Output Value")
    }
    ... (Many more field definitions
    }
}
\end{verbatim}The include file, \verb|aoRecord|.\verb|h|, generated by \verb|dbToRecordtypeH| contains:

\begin{verbatim}#include "ellLib.h"
#include "epicsMutex.h"
#include "link.h"
#include "epicsTime.h"
#include "epicsTypes.h"

#ifndef INCaoOIFH
#define INCaoOIFH
typedef enum {
        aoOIF_Full,
        aoOIF_Incremental,
}aoOIF;
#endif /*INCaoOIFH*/
#ifndef INCaoH
#define INCaoH
typedef struct aoRecord {
        char            name[29]; /*Record Name*/
        ... Remaining fields in database common
        double          val;    /*Desired Output*/
        double          oval;   /*Output Value*/
        ... remaining record specific fields
} aoRecord;
#define aoRecordNAME    0
... defines for remaining fields in database common
#define aoRecordVAL     42
#define aoRecordOVAL    43
... defines for remaining record specific fields
#ifdef GEN_SIZE_OFFSET
int aoRecordSizeOffset(dbRecordType *pdbRecordType)
{
    aoRecord *prec = 0;
  pdbRecordType->papFldDes[0]->size=sizeof(prec->name);
  pdbRecordType->papFldDes[0]->offset=
(short)((char *)&prec->name - (char *)prec);
  ... code to compute size&offset for other fields in dbCommon
  pdbRecordType->papFldDes[42]->size=sizeof(prec->val);
  pdbRecordType->papFldDes[42]->offset=
(short)((char *)&prec->val - (char *)prec);
  pdbRecordType->papFldDes[43]->size=sizeof(prec->oval);
  pdbRecordType->papFldDes[43]->offset=
(short)((char *)&prec->oval - (char *)prec);
  ... code to compute size&offset for remaining fields
  pdbRecordType->rec_size = sizeof(*prec);
  return(0);
}
#endif /*GEN_SIZE_OFFSET*/

\end{verbatim}The analog output record support module and all associated device support modules should use this include file. No other 
code should use it. Let's discuss the various parts of the file.:

\begin{itemize}\item The \verb|enum| generated from the menu definition should be used to reference the value of the field associated with the 
menu.

\item The \verb|typedef| and \verb|structure| defining the record are used by record support and device support to access fields 
in an analog output record.

\item A \verb|#define| is present for each field within the record. This is useful for the record support routines that are passed 
a pointer to a \verb|DBADDR| structure. They can have code like the following:

\end{itemize}\begin{verbatim}
switch (dbGetFieldIndex(pdbAddr)) {
    case aoRecordVAL :
        ...
        break;
    case aoRecordXXX:
        ...
        break;
    default:
        ...
}
\end{verbatim}The C source routine \verb|aoRecordSizeOffset| is automatically called when a record type file is loaded into an IOC. 
Thus user code does not have to be aware of this routine except for the following convention: The associate record 
support module MUST include the statements:

\begin{verbatim}#define GEN_SIZE_OFFSET
#include "xxxRecord.h"
#undef GEN_SIZE_OFFSET
\end{verbatim}This convention ensures that the routine is defined exactly once.

\section{dbExpand}

\index{dbExpand}\begin{verbatim}dbExpand -Idir -Smacsub -ooutfile file1 file2 ...
\end{verbatim}Multiple \verb|-I| options can be specified for an include path, and multiple \verb|-S| options for macro substitution. If no output 
filename is specified with \verb|-ooutfile| then the output will go to stdout. Note that the environment variable 
\verb|EPICS_DB_INCLUDE_PATH| can also be used in place of the \verb|-I| options.

NOTE: This is supported only on the host.

This command reads all the input files and then writes a file containing the definitions for all information described by the 
input files. The output content differs from the input in that comment lines do not appear and all include files are 
expanded.

This routine is extremely useful if an IOC is not using NFS for the \verb|dbLoadDatabase| commands. It takes more than 2 
minutes to load the \verb|base/rec/base.dbd| file into an IOC if NFS is not used. If \verb|dbExpand| creates a local \verb|base.dbd| 
file, it takes about 7 seconds to load (25 MHZ 68040 IOC).

\section{dbLoadDatabase}

\index{dbLoadDatabase}\begin{verbatim}dbLoadDatabase(char *dbdfile, char *path, char *substitutions)
\end{verbatim}NOTES: 

\begin{itemize}\item IOC Only

\item Using a path on a vxWorks ioc does not work very well.

\item Both path and substitutions can be null.

\item \verb|dbdfile| may contain environment variable macros of the form \verb|${MOTOR}| which will be expanded before the 
file is opened.

\end{itemize}This command loads a database file containing any of the definitions given in the summary at the beginning of this 
chapter. Note that normally \verb|dbLoadDatabase| is only used to load a Database Definition (.dbd) file, but it is currently 
possible to use it for loading Record Instance (.db) files as well.

\verb|As each line of dbdfile| is read, the substitutions specified in \verb|substitutions| is performed\verb|. |The 
substitutions are specified as follows:

\begin{verbatim}"var1=sub1,var2=sub3,..."
\end{verbatim}Variables are specified in the dbfile as \$(variable\_name). If the substitution string

\begin{verbatim}"a=1,b=2,c=\"this is a test\""
\end{verbatim}were used, any variables \verb|$|(\verb|a|), \verb|$|(\verb|b|), \verb|$|(\verb|c|) would be substituted with the appropriate data.

\section{dbLoadRecords}

\index{dbLoadRecords}\begin{verbatim}dbLoadRecords(char* dbfile, char* substitutions)
\end{verbatim}NOTES:

\begin{itemize}\item IOC Only.

\item The file indicated by \verb|dbfile| should contain only record instances, record aliases and/or breakpoint tables.

\item The \verb|dbfile| string may itself contain environment variable macros of the form \verb|${MOTOR}| which will be 
expanded before the file is opened.

\end{itemize}\subsection{EXAMPLE}

For example, let \verb|test|.\verb|db| be:

\begin{verbatim}
record(ai,"$(pre)testrec1")
record(ai,"$(pre)testrec2")
record(stringout,"$(pre)testrec3") {
    field(VAL,"$(STR)")
    field(SCAN,"$(SCAN)")
}
\end{verbatim}Then issuing the command:

\begin{verbatim}dbLoadRecords("test.db",0,"pre=TEST,STR=test,SCAN=Passive")
\end{verbatim}gives the same results as loading:

\begin{verbatim}record(ai,"TESTtestrec1")
record(ai,"TESTtestrec2")
record(stringout,"TESTtestrec3") {
    field(VAL,"test")
    field(SCAN,"Passive")
}
\end{verbatim}

\section{dbLoadTemplate}

\index{dbLoadTemplate}\begin{verbatim}dbLoadTemplate(char* template_def)
\end{verbatim}NOTES:

\begin{itemize}\item IOC Only.

\item MSI can be used to expand templates on the host instead of using this.

\end{itemize}\verb|dbLoadTemplate| reads a template definition file. This file contains rules about loading database instance files, which 
contain \verb|$|(\verb|xxx|) macros, and performing substitutions.

\verb|template_def| contains the rules for performing substitutions on the instance files. For convenience two formats are 
provided. The format is either:

\begin{verbatim}file name.template {
    { var1=sub1_for_set1, var2=sub2_for_set1, var3=sub3_for_set1, ... }
    { var1=sub1_for_set2, var2=sub2_for_set2, var3=sub3_for_set2, ... }
    { var1=sub1_for_set3, var2=sub2_for_set3, var3=sub3_for_set3, ... }
}
\end{verbatim}or:

\begin{verbatim}file name.template {
pattern { var1, var2, var3, ... }
    { sub1_for_set1, sub2_for_set1, sub3_for_set1, ... }
    { sub1_for_set2, sub2_for_set2, sub3_for_set2, ... }
    { sub1_for_set3, sub2_for_set3, sub3_for_set3, ... }
}
\end{verbatim}The first line (\verb|file name.template|) specifies the record instance input file. The file name may appear inside double 
quotation marks; these are required if the name contains any characters that are not in the following set, or if it contains 
environment variable macros of the form \verb|${ENV_VAR_NAME}| which are to be expanded before the file is opened:

\begin{verbatim}a-z A-Z 0-9 _ + - . / \ : ; [ ] < >
\end{verbatim}Each set of definitions enclosed in \verb|{}| is variable substitution for the input file. The input file has each set applied to it to 
produce one composite file with all the completed substitutions in it. Version 1 should be obvious. In version 2, the 
variables are listed in the ``\verb|pattern|\{\}" line, which must precede the braced substitution lines. The braced substitution 
lines contains sets which match up with the \verb|pattern|\{\} line.

\subsection{EXAMPLE}

Two simple template file examples are shown below. The examples specify the same substitutions to perform: 
\verb|this|=\verb|sub1| and \verb|that|=\verb|sub2| for a first set, and \verb|this|=\verb|sub3| and \verb|that|=\verb|sub4| for a second set. 

\begin{verbatim}
file test.template {
    { this=sub1,that=sub2 }
    { this=sub3,that=sub4 }
}

file test.template {
    pattern{this,that}
    {sub1,sub2}
    {sub3,sub4 }
}
\end{verbatim}Assume that the file \verb|test|.\verb|template| contains:

\begin{verbatim}
record(ai,"$(this)record") {
    field(DESC,"this = $(this)")
}
record(ai,"$(that)record") {
    field(DESC,"this = $(that)")
}
\end{verbatim}

Using \verb|dbLoadTemplate| with either input is the same as defining the records:

\begin{verbatim}record(ai,"sub1record") {
    field(DESC,"this = sub1")
}
record(ai,"sub2record") {
    field(DESC,"this = sub2")
}

record(ai,"sub3record") {
    field(DESC,"this = sub3")
}
record(ai,"sub4record") {
    field(DESC,"this = sub4")
}
\end{verbatim}\section{dbReadTest}

\index{dbReadTest}\begin{verbatim}dbReadTest -Idir -Smacsub file.dbd ... file.db ...
\end{verbatim}This utility can be used to check for correct syntax in database definition and database instance files. It just reads all the 
specified files

Multiple \verb|-I,| and \verb|-S |options can be specified. An arbitrary number of database definition and database instance files can 
be specified.
