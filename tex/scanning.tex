\chapter{Database Scanning}

\section{Overview}

Database scanning is the mechanism for deciding when to process a record.
Five types of scanning are possible:

\begin{itemize}
\index{Periodic - Scan Type}
\item Periodic: A record can be processed periodically.
A number of standard time intervals are supported and additional periods can be added.

\index{Event - Scan Type}
\item Event: Event scanning is based on the posting of a named or numbered event by another component of the software.

\index{I/O Event - Scan Type}
\item I/O Event: The original meaning of this scan type is a request for record processing as a result of a hardware interrupt.
The mechanism supports hardware interrupts as well as software generated events.

\index{Passive - Scan Type}
\item Passive: Passive records are processed only via requests to \verb|dbScanPassive|.
This happens when database links (Forward, Input, or Output), which have been declared ``Process Passive" are accessed during record processing.
It can also happen as a result of \verb|dbPutField| being called (which normally results from a Channel Access put request).

\index{Scan Once - Scan Type}
\item Scan Once: In order to provide for caching puts, the scanning system provides a routine \verb|scanOnce| which arranges for a record to be processed one time.

\end{itemize}

This chapter explains database scanning in increasing order of detail.
It first explains database fields involved with scanning.
It next discusses the interface to the scanning system.
The last section gives a brief overview of how the scanners are implemented.

\section{Scan Related Database Fields}

\index{Scan Related Database Fields}
The following fields are normally set from within a database configuration tool.
It is quite permissible however to change any of these scan-related fields of a record dynamically.
For example, a display manager screen could tie a menu control to the \verb|SCAN| field of a record and allow the operator to dynamically change the scan mechanism.

\subsection{SCAN}

\index{SCAN - Scan Related Field}
This field, which specifies the scan mechanism, has an associated menu with the following choices:

\begin{description}
\index{Passive}
\item \verb|Passive| - Passively scanned.

\index{Event}
\item \verb|Event| - Event Scanned. The field \verb|EVNT| specifies the event name or number.

\index{I/O Event scanned}
\item \verb|I/O Intr| - I/O Event scanned.

\item \verb|10 Second| - Periodically scanned every 10 seconds

\item ...

\item \verb|.1 Second| - Periodically scanned every .1 seconds

\end{description}

\subsection{PHAS - Scan Phase}

\index{PHAS - Scan Related Field}
This 16-bit integer field determines relative processing order for records that are in the same scan set.
For example all records periodically scanned at a 2 second rate belong to the same scan set.
All Event scanned records with the same \verb|EVNT| belong to the same scan set, etc.
For records in the same scan set, all records with \verb|PHAS|=0 are processed before records with \verb|PHAS|=1, which are processed before all records with \verb|PHAS|=2, etc.

In general it is not a good idea to rely on \verb|PHAS| to enforce processing order.
It is better to use database links.

\subsection{EVNT - Named or Numbered Events}

\index{EVNT - Scan Related Field}
This field is only used when \verb|SCAN| is set to \verb|Event|, when \verb|EVNT| specifies the associated database event name or number.
For named events the \verb|EVNT| field should be set to the event name.
Event names are compared using \verb|strcmp()|, so case and leading/trailing spaces must all match.
To use the numeric event trigger routine \verb|post_event()| the \verb|EVNT| field must hold an integer in the range 1...255.

\subsection{PRIO - Scheduling Priority }

\index{PRIO - Scan Related Field}
This field can be used by any software component that needs to specify a scheduling priority.
The Event and I/O event scan types use this field.

\section{Scan Related Software Components}

\subsection{menuScan.dbd}

\index{menuScan.dbd}
This file holds the definition of the menu used by the field \verb|SCAN|.
The default definition is:

\begin{verbatim}
menu(menuScan) {
    choice(menuScanPassive,"Passive")
    choice(menuScanEvent,"Event")
    choice(menuScanI_O_Intr,"I/O Intr")
    choice(menuScan10_second,"10 second")
    choice(menuScan5_second,"5 second")
    choice(menuScan2_second,"2 second")
    choice(menuScan1_second,"1 second")
    choice(menuScan_5_second,".5 second")
    choice(menuScan_2_second,".2 second")
    choice(menuScan_1_second,".1 second")
}
\end{verbatim}

The first three choices must appear in the order and location shown.
The remaining definitions are for the periodic scan rates, which should appear in the order slowest to fastest (the order directly controls the thread priority assigned to the particular scan rate, and faster scan rates should be assigned higher thread priorities).
At IOC initialization, the menu choice strings are read while the scan system is being initialized.
The number of periodic scan rates and the period of each rate is determined from the menu choice strings.
Thus periodic scan rates can be changed by copying \verb|menuScan.dbd| into the IOC's build directory and modifying the set of choices defined therein.
The choice names such as \verb|menuScan10_second| are not used in this case, but must still be unique.
Each periodic choice string must begin with a number and be followed by any of the following unit strings:

\begin{description}
\item \verb|second| or \verb|seconds|
\item \verb|minute| or \verb|minutes|
\item \verb|hour| or \verb|hours|
\item \verb|Hz| or \verb|Hertz|
\end{description}

\index{dbScan.h}
\subsection{dbScan.h}

All software components that interact with the scanning system must include this file.

The most important definitions in this file are:

\begin{verbatim}
#define SCAN_PASSIVE       menuScanPassive
#define SCAN_EVENT         menuScanEvent
#define SCAN_IO_EVENT      menuScanI_O_Intr
#define SCAN_1ST_PERIODIC  (menuScanI_O_Intr + 1)

typedef struct ioscan_head *IOSCANPVT;
typedef struct event_list *EVENTPVT;
typedef void (*io_scan_complete)(void *usr, IOSCANPVT, int prio);
typedef void (*once_complete)(void *usr, struct dbCommon*);

long scanInit(void);
void scanRun(void);
void scanPause(void);
void scanShutdown(void);

EVENTPVT eventNameToHandle(const char* event);
void postEvent(EVENTPVT epvt) EPICS_DEPRECATED;
void post_event(int event);
void scanAdd(struct dbCommon *);
void scanDelete(struct dbCommon *);
double scanPeriod(int scan);
void scanOnce(struct dbCommon *precord);
int scanOnce3(struct dbCommon *, once_complete cb, void *usr);
int scanOnceSetQueueSize(int size);

/*print periodic lists*/
epicsShareFunc int scanppl(double rate);

/*print event lists*/
epicsShareFunc int scanpel(const char *event_name);

/*print io_event list*/
epicsShareFunc int scanpiol(void);

void scanIoInit(IOSCANPVT *ppios);
unsigned int scanIoImmediate(IOSCANPVT pios, int prio);
unsigned int scanIoRequest(IOSCANPVT pios);
void scanIoSetComplete(IOSCANPVT, io_scan_complete, void *usr);
\end{verbatim}

\index{SCAN\_1ST\_PERIODIC}
The first set of definitions defines the various scan types.
The typedefs are used when interfacing with the routines below.
The remaining definitions declare the public scan access routines.
These are described in the following subsections.

\subsection{Initializing And Controlling Database Scaning}

\begin{verbatim}
long scanInit(void);
\end{verbatim}

\index{scanInit}
The routine \verb|scanInit| is called by \verb|iocInit|.
It initializes the scanning system.

\index{scanRun}
\index{scanPause}
\index{scanShutdown}
\begin{verbatim}
void scanRun(void);
void scanPause(void);
void scanShutdown(void);
\end{verbatim}

These routines start, pause and stop all the scan tasks respectively.
They are used by the \verb|iocInit|, \verb|iocRun|, \verb|iocPause| and \verb|iocShutdown| commands.

\subsection{Adding And Deleting Records From Scan List}

The following routines are called each time a record is added to or deleted from a scan list.

\index{scanAdd}
\index{scanDelete}
\begin{verbatim}
scanAdd(struct dbCommon *);
scanDelete(struct dbCommon *);
\end{verbatim}

These routines are called by \verb|scanInit| at IOC initialization time in order to enter all records into the correct scan list.
The routine \verb|dbPut| calls \verb|scanDelete| and \verb|scanAdd| each time a scan-related field is changed (scan-related fields are declared to be \verb|SPC_SCAN| in \verb|dbCommon.dbd|).
\verb|scanDelete| is called before the field is modified and \verb|scanAdd| after the field is modified.

\subsection{Obtaining the scan period from the SCAN field}

\begin{verbatim}
double scanPeriod(int scan);
\end{verbatim}

\index{scanPeriod}
The argument is the index into the set of enum choices from menuScan.dbd.
Most users will pick the value from the \verb|SCAN| field of a database record.
The routine returns the scan period in seconds.
The result will be 0.0 if scan doesn't refer to a periodic scan rate.

\subsection{Declaring and Triggering Database Events}

Any software component may declare and subsequently trigger a database event.
Database events used to be numbered with 8-bit integers and did not have to be declared in advance.
Since Base 3.15 though events can now be named, in which case they must be declared to convert the name into an event object.

\index{eventNameToHandle}
\begin{verbatim}
EVENTPVT eventNameToHandle(const char* event);
\end{verbatim}

This routine must be called from task context (i.e. not from an interrupt service routine) to convert an event's name into an associated \verb|EVENTPVT| handle.
The first time each name is seen a handle will be created for it; subsequent calls to \verb|eventNameToHandle| with the same name will return the same handle.

A database event is triggered by calling one of:

\index{postEvent}
\index{post\_event}
\begin{verbatim}
void postEvent(EVENTPVT eventObj);
void post_event(int eventNum) EPICS_DEPRECATED;
\end{verbatim}

The original integer \verb|post_event| routine is now deprecated in favor of the new routine \verb|postEvent| that takes an event handle instead of the event number.
These event-posting routines may be called by virtually any IOC software component, including from an interrupt service routine on VxWorks or RTEMS.
For example sequence programs can call them.
The record support module for the \verb|eventRecord| calls \verb|postEvent|.

\subsection{Interfacing to I/O Event Scanning}

\index{I/O Event Scanning}

Interfacing to the I/O event scanner is done via some combination of device and driver support.

\begin{enumerate}
\item Include \verb|dbScan.h|

\item For each separate I/O event source the following must be done:

\begin{enumerate}

\item Declare an \verb|IOSCANPVT| variable, e.g.

\begin{verbatim}
static IOSCANPVT ioscanpvt;
\end{verbatim}

\item Call \verb|scanIoInit| during initialization, e.g.

\begin{verbatim}
scanIoInit(&ioscanpvt);
\end{verbatim}
\end{enumerate}

\item Provide the device support \verb|get_ioint_info| routine.
This routine has the prototype:

\begin{verbatim}
long get_ioint_info(int cmd, struct dbCommon *precord,
    IOSCANPVT *ppvt);
\end{verbatim}

This routine will be called each time the record pointed to by \verb|precord| is added to or deleted from an I/O Event scan list.
The \verb|cmd| argument will be zero if the record is being added to an I/O event list, 1 if it is being deleted from the list.
This routine must set \verb|*ppvt| to the \verb|IOSCANPVT| variable associated with this record.

\item Whenever an I/O event is detected, the device software must call \verb|scanIoRequest| or \verb|scanIoImmediate|, e.g.

\begin{verbatim}
scanIoRequest(ioscanpvt);
scanIoImmediate(ioscanpvt, priorityLow);
\end{verbatim}

The routine \verb|scanIoRequest()| may be safely called from interrupt level.
A request is queued and will be handled by one of the standard callback threads.
There are three sets of callback threads fed from three queues, one for each priority level (see section \ref{sec:callbackThreads}); the \verb|PRIO| field of a record determines which queue will be used for processing this record after \verb|scanIoRequest()| has been called.

The routine \verb|scanIoImmediate()| may not be called from interrupt level.
Instead of queuing a request, this routine directly processes records on the current thread.
Note that it only scans records with a certain priority level,
and must therefore be called for each priority level.

\verb|scanIoRequest()| will return a bit pattern indicating to which queues the request was added.
A return value of zero means no records are currently configured to use this interrupt source for I/O Interrupt scanning.

\item Device or driver support that needs to implement flow control can set up a completion callback by calling \verb|scanIoSetComplete|, e.g.

\begin{verbatim}
static void myCallback(void *arg, IOSCANPVT pvt, int prio) {
    ...
}

scanIoSetComplete(ioscanpvt, myCallback, (void *)arg);
\end{verbatim}

The completion callback will be run from one of the callback threads, once per priority actually used (bits set in the return value of \verb|scanIoRequest|), after the list of records with that priority level has been processed.
Note that for records with asynchronous device support, record processing might not have completed when the callback is run.
\end{enumerate}

The following code fragment shows an event record device support module that supports I/O event scanning: 

\begin{verbatim}
#include  <vxWorks.h>
#include  <types.h>
#include  <stdioLib.h>
#include  <intLib.h>
#include  <dbDefs.h>
#include  <dbAccess.h>
#include  <dbScan.h>
#include  <recSup.h>
#include  <devSup.h>
#include  <eventRecord.h>
/* Create the dset for devEventXXX */
long init();
long get_ioint_info();
struct {
    long  number;
    DEVSUPFUN  report;
    DEVSUPFUN  init;
    DEVSUPFUN  init_record;
    DEVSUPFUN  get_ioint_info;
    DEVSUPFUN  read_event;
}devEventTestIoEvent={
    5,
    NULL,
    init,
    NULL,
    get_ioint_info,
    NULL};
static IOSCANPVT ioscanpvt;
static void int_service(IOSCANPVT ioscanpvt)
{
    scanIoRequest(ioscanpvt);
}

static long init()
{
    scanIoInit(&ioscanpvt);
    intConnect(<vector>,(FUNCPTR)int_service,ioscanpvt);
    return(0);
}
static long get_ioint_info(
int   cmd,
struct eventRecord   *pr,
IOSCANPVT   *ppvt)
{
    *ppvt = ioscanpvt;
    return(0);
}
\end{verbatim}

\index{get\_ioint\_info}
\section{Implementation Overview}

The code for the entire scanning system resides in \verb|dbScan.c|.
This section gives an overview of how this code is organized.

\subsection{Definitions And Routines Common To All Scan Types}

Everything is built around two basic structures:

\begin{verbatim}
typedef struct scan_list {
    epicsMutexId lock;
    ELLLIST      list;
    short        modified;
};

typedef struct scan_element{
    ELLNODE         node;
    scan_list       *pscan_list;
    struct dbCommon *precord;
}
\end{verbatim}

Each \verb|scan_list.list| is the head of a list of \verb|scan_element| nodes pointing to records that all belong to the same scan set.
For example, all records that are periodically scanned at the 1 second rate are in the same scan set.
The libCom \verb|ellLib| routines are used to access the list.
The \verb|scan_element.node| field contains the next and previous links.
Each record that appears in a \verb|scan_list| has an associated \verb|scan_element|.
The \verb|SPVT| field which appears in \verb|dbCommon| points to the associated \verb|scan_element|.

The \verb|lock|, \verb|modified|, and \verb|pscan_list| fields allow \verb|scan_elements|, i.e. records, to be dynamically removed and added to scan lists.
If \verb|scanList|, the routine which actually processes a scan list, is studied it can be seen that these fields allow the list to be scanned very efficiently when no modifications are made to the list while it is being scanned.
This is, of course, the normal case.

The \verb|dbScan.c| module contains several private routines.
The following access a single scan set: 

\begin{itemize}
\item \verb|printList| - Prints the names of all records in a scan set.

\item \verb|scanList| - This routine is the heart of the scanning system.
For each record in a scan set it does the following:

\begin{verbatim}
dbScanLock(precord);
dbProcess(precord);
dbScanUnlock(precord);
\end{verbatim}

It also has code to recognize when a scan list is modified while the scan set is being processed.

\item \verb|addToList| - This routine adds a new element to a scan list.

\item \verb|deleteFromList| - This routine deletes an element from a scan list.

\end{itemize}

\subsection{Database Event Scanning}

\index{Database Event Scanning}
\index{Event Scanning}
Event scanning is built around the following definitions:

\begin{verbatim}
typedef struct event_list {
    CALLBACK           callback[NUM_CALLBACK_PRIORITIES];
    scan_list          scan_list[NUM_CALLBACK_PRIORITIES];
    struct event_list *next;
    char               event_name[MAX_STRING_SIZE];
} event_list;

static event_list * volatile pevent_list[256];
static epicsMutexId event_lock;
\end{verbatim}

Event scanning uses the general purpose callback tasks to perform record processing, i.e. no extra threads are spawned for this.
When a named event is declared by a call to \verb|eventNameToHandle()| an \verb|event_list| will be created for that named event.
Every \verb|event_list| contains a \verb|scan_list| for each of the 3 priorities.
The \verb|next| member is used to keep a singly-linked list of all the \verb|event_list| objects, with the first item on that list pointed to by \verb|pevent_list[0]|.
\verb|pevent_list| is an array of pointers to numbered \verb|event_list| objects, and is used when an event name is an integer in the range 1..255.
It provides fast access to 255 numbered events, i.e. one for each possible numeric database event.

\subsubsection{postEvent}

\index{postEvent}
\begin{verbatim}
void postEvent(event_list *pel);
\end{verbatim}

This routine is called to request an event scan for a named event handle.
It may be called from interrupt level.
It looks at each \verb|scan_list| in the \verb|event_list| (one for each callback priority) and if any nodes are present in the list it makes a \verb|callbackRequest| to process that set of records.
The appropriate callback task calls routine \verb|eventCallback|, which just calls \verb|scanList|.

\subsubsection{post\_event}

\index{post\_event}
\begin{verbatim}
void post_event(int eventNum) EPICS_DEPRECATED;
\end{verbatim}

This routine is called to request an event scan for a numbered event.
It may be called from interrupt level.
It looks up the \verb|event_list| indicated by the given event number and calls \verb|postEvent| with that handle.

\subsection{I/O Event Scanning}

\index{I/O Event Scanning}
I/O event scanning is built around the following definitions:

\begin{verbatim}
typedef struct io_scan_list {
    CALLBACK callback;
    scan_list scan_list;
} io_scan_list;

typedef struct ioscan_head {
    struct ioscan_head *next;
    struct io_scan_list iosl[NUM_CALLBACK_PRIORITIES];
    io_scan_complete cb;
    void *arg;
} ioscan_head;

static ioscan_head *pioscan_list = NULL;
static epicsMutexId ioscan_lock;
\end{verbatim}

I/O event scanning uses the general purpose callback tasks to perform record processing, i.e. no extra threads are spawned for this.
The callback field of \verb|io_scan_list| is used to communicate with the callback tasks.

The following routines implement I/O event scanning:

\subsubsection{scanIoInit}

\index{scanIoInit}
\begin{verbatim}
void scanIoInit(IOSCANPVT *ppios)
\end{verbatim}

This routine is called by device or driver support.
It must be called once for each interrupt source.
\verb|scanIoInit| allocates and initializes an \verb|ioscan_head| object which contains an \verb|io_scan_list| for each callback priority.
It puts the address of the allocated object in \verb|ppios|.

When \verb|scanAdd| or \verb|scanDelete| are called, they call the device support routine \verb|get_ioint_info| which returns \verb|ppios|.
The \verb|scan_element| is then added to or deleted from the correct scan list.

\subsubsection{scanIoRequest}

\index{scanIoRequest}
\begin{verbatim}
unsigned int scanIoRequest(IOSCANPVT pios)
\end{verbatim}

This routine is called by device or driver support to request a specific I/O event scan.
It may be called from interrupt level.
It looks at each \verb|io_scan_list| referenced by \verb|pios| (one for each callback priority) and if any elements are present in the \verb|scan_list| a \verb|callbackRequest| is issued.
The appropriate callback task calls routine \verb|ioscanCallback|, which calls \verb|scanList| followed by any completion callback that was registered with \verb|pios|.

The \verb|scan_element| is then added to or deleted from the correct scan list.

\subsubsection{scanIoImmediate}

\index{scanIoRequest}
\begin{verbatim}
unsigned int scanIoImmediate(IOSCANPVT pios, int prio)
\end{verbatim}

Scans any records in the given IOSCANPVT with the given priority.
Processing is done on the current thread.
Allows device or driver support to implement private scanning threads.

Such device or driver support should call \verb|scanIoImmediate| for all priority levels.
By convention this should be done concurrently.

\subsection{Periodic Scanning}

\index{Periodic Scanning}
Periodic scanning is built around the following definitions:

\begin{verbatim}
typedef struct periodic_scan_list {
    scan_list           scan_list;
    double              period;
    const char          *name;
    unsigned long       overruns;
    volatile enum ctl   scanCtl;
    epicsEventId        loopEvent;
} periodic_scan_list;

static int nPeriodic;
static periodic_scan_list **papPeriodic;
static epicsThreadId *periodicTaskId;
\end{verbatim}

The \verb|nPeriodic| value is the number of periodic scan rates in use, determined at \verb|iocInit| time by the \verb|initPeriodic| routine.
\verb|papPeriodic| points to an array of pointers to \verb|periodic_scan_lists|.
There is an array element for each scan rate.
A periodic scan task is created for each scan rate.

The following routines implement periodic scanning:

\subsubsection{initPeriodic}

\index{initPeriodic}
\begin{verbatim}
void initPeriodic(void)
\end{verbatim}

This routine first determines how many periodic scan rates are to be created from the definition of the \verb|menuScan| menu.
The array of pointers referenced by \verb|papPeriodic| is allocated.
For menu choice a \verb|periodic_scan_list| is allocated and initialized.
It parses the choice string for that choice to obtain the scan period for the scan.

\subsubsection{periodicTask}

\index{periodicTask}
\begin{verbatim}
periodicTask (struct scan_list *psl)
\end{verbatim}

In outline this task runs an infinite loop, calling \verb|scanList| and then waiting until the start of the next scan interval, allowing for the time it took to scan the list.
If a periodic scan list takes longer to process than its defined scan period, its next scan will be delayed by half a scan period, with a maximum of 1 second delay.
This does not limit what scan rates can actually be implemented, as long as all the records in the list can be processed within the requested period.
Persistent over-runs (more than 10 times in a row) will result in a warning message being logged.
The total number of over-runs is counted by each scan thread and can be displayed using the \verb|scanppl| command.

\subsection{Scan Once}

\subsubsection{scanOnce}

\index{scanOnce}
\begin{verbatim}
typedef void (*once_complete)(void *usr, struct dbCommon*);
int scanOnce (dbCommon *precord)
int scanOnce3(struct dbCommon *, once_complete cb, void *usr)
\end{verbatim}

A task \verb|onceTask| waits for requests to issue a \verb|dbProcess| request.
The routine \verb|scanOnce| puts the address of the record to be processed in a ring buffer and wakes up \verb|onceTask|.

This routine may be called from interrupt level.

The \verb|scanOnce3| variant also takes a callback function and user pointer which are invoked from the \verb|onceTask|
after the record is processed.

These functions return zero when a request is successfull queued, and a non-zero error code if a request can't be queued.

\subsubsection{SetQueueSize}

\verb|scanOnce| places its requests into a ring buffer.
This is set by default to be 1000 entries long.
The size can be changed by executing the following command in the startup script before \verb|iocInit|:

\index{scanOnceSetQueueSize}
\begin{verbatim}
int scanOnceSetQueueSize(int size);
\end{verbatim}

