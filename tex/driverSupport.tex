\chapter{Driver Support}\index{Driver Support}

\section{Overview}

It is not necessary to create a driver support module in order to interface EPICS to hardware. For simple hardware device 
support is sufficient. At the present time most hardware support has both. The reason for this is historical. Before EPICS 
there was GTACS. During the change from GTACS to EPICS, record support was changed drastically. In order to 
preserve all existing hardware support the GTACS drivers were used without change. The device support layer was 
created just to shield the existing drivers form the record support changes.

Since EPICS now has both device and driver support the question arises: When do I need driver support and when don't 
I? Lets give a few reasons why drivers should be created.

\begin{itemize}\item The hardware is actually a subnet, e.g. GPIB. In this case a driver should be provided for accessing the subnet. 
There is no reason to make the driver aware of EPICS except possibly for issuing error messages.

\item The hardware is complicated. In this case supplying driver support helps modularized the software. The Allen 
Bradley driver, which is also an example of supporting a subnet, is a good example.

\item An existing driver, maintained by others, is available. I don't know of any examples.

\item The driver should be general purpose, i.e. not tied to EPICS. The CAMAC driver is a good example. It is used by 
other systems, such as CODA. This is perhaps the most important reason for driver support.

\item For common devices, e.g. GPIB, CAN, CAMAC, etc. a generic driver layer should be created. This generic layer 
should be independent of EPICS and independent of low level interfaces. It should also define an inteface for low 
level drivers. This allows low level interfaces to be replaced without impacting IOC records, record support, or 
device support.

\end{itemize}The only thing needed to interface a driver to EPICS is to provide a driver support module, which can be layered on top of 
an existing driver, and provide a database definition for the driver. The driver support module is described in the next 
section. The database definition is described in chapter ``Database Definition".

\section{Device Drivers}

Device drivers are modules that interface directly with the hardware. They are provided to isolate device support routines 
from details of how to interface to the hardware. Device drivers have no knowledge of the internals of database records. 
Thus there is no necessary correspondence between record types and device drivers. For example the Allen Bradley driver 
provides support for many different types of signals including analog inputs, analog outputs, binary inputs, and binary 
outputs. 

In general only device support routines know how to call device drivers. Since device support varies widely from device 
to device, the set of routines provided by a device driver is almost completely driver dependent. The only requirement is 
that routines \verb|report| and \verb|init| must be provided. Device support routines must, of course, know what routines are 
provided by a driver.

File \verb|drvSup.h| describes the format of a driver support entry table. The driver support module must supply a driver entry 
table. An example definition is:

\begin{verbatim}LOCAL long report();
LOCAL long init();
struct {
        long    number;
        DRVSUPFUN       report;
        DRVSUPFUN       init;
} drvAb={
        2,
        report,
        init
};
epicsExportAddress(drvet,drvGpib);
\end{verbatim}\index{Driver Support Entry Table Example}The above example is for the Allen Bradley driver. It has an associated ascii definition of:

\begin{verbatim}driver(drvGpib)
\end{verbatim}Thus it is seen that the driver support module should supply two EPICS callable routines: \verb|int| and \verb|report|.

\subsubsection{init}

This routine, which has no arguments, is called by \verb|iocInit|. The driver is expected to look for and initialize the 
hardware it supports. As an example the init routine for Allen Bradley is:

\begin{verbatim}LOCAL long init()
{
    return(ab_driver_init());
}
\end{verbatim}\subsubsection{report}

The report routine is called by the \verb|dbior|, an IOC test routine. It is responsible for producing a report describing the 
hardware it found at init time. It is passed one argument, level, which is a hint about how much information to display. An 
example, taken from Allen Bradley, is:

\begin{verbatim}LOCAL long report(int level)
{
   return(ab_io_report(level));
}
\end{verbatim}Guidelines for level are as follows:

\begin{description}\item Level=0 Display a one line summary for each device

\item Level=1 Display more information

\item Level=2 Display a lot of information. It is even permissible to prompt for what is wanted.

\end{description}\subsubsection{Hardware Configuration}

Hardware configuration includes the following:

\begin{itemize}\item VME/VXI address space

\item VME Interrupt Vectors and levels

\item Device Specific Information

\end{itemize}The information contained in hardware links supplies some but not all configuration information. In particular it does not 
define the VME/VXI addresses and interrupt vectors. This additional information is what is meant by hardware 
configuration in this chapter.

The problem of defining hardware configuration information is an unsolved problem for EPICS. At one time 
configuration information was defined in \verb|module_types|.\verb|h| Many existing device/driver support modules still uses this 
method. It should NOT be used for any new support for the following reasons:

\begin{itemize}\item There is no way to manage this file for the entire EPICS community.

\item It does not allow arbitrary configuration information.

\item It is hard for users to determine what the configuration information is.

\end{itemize}The fact that it is now easy to include ASCII definitions for only the device/driver support used in each IOC makes the 
configuration problem much more manageable than previously. Previously if you wanted to support a new VME modules 
it was necessary to pick addresses that nothing in \verb|module_types|.\verb|h| was using. Now you only have to check modules 
you are actually using.

Since there are no EPICS defined rules for hardware configuration, the following minimal guidelines should be used:

\begin{itemize}

\item Never use \verb|#define| to specify things like VME addresses. Instead use variables and assign default values. Allow 
the default values to be changed before iocInit is executed. The best way is to supply a global routine that can be 
invoked from the IOC startup file. Note that all arguments to such routines should be one of the following:

\begin{description} 

\item \verb|int|

\item \verb|char *|

\item \verb|double|

\end{description}

\item Call the routines described in chapter ``Device Support Library" whenever possible.

\end{itemize}






