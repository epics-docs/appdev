\chapter{IOC Shell}
\label{chap:IOC Shell}
\index{IOC Shell}

\section{Introduction}

The EPICS IOC shell is a simple command interpreter which provides a subset of the capabilities of the vxWorks shell.
It is used to interpret startup scripts (st.cmd) and to execute commands entered at the console terminal.
In most cases vxWorks startup scripts can be interpreted by the IOC shell without modification.
The following sections of this chapter describe the operation of the IOC shell from the user's and programmer's points of view.

\section{IOC Shell Operation}

The IOC shell reads lines of input, expands environment variable parameters, breaks the line into commands and arguments then calls functions corresponding to the decoded command.
Commands and arguments are separated by one or more `space' characters.
Characters interpreted as spaces include the actual space character and the tab character as well as commas and open and close parentheses.
Thus, the command line

\begin{verbatim}
dbLoadRecords("db/dbExample1.db","user=mrk")
\end{verbatim}

would be interpreted by the IOC shell as the \verb|dbLoadRecords| command with arguments \verb|db/dbExample1.db| and \verb|user=mrk|.

Unrecognized commands result in a diagnostic message but are otherwise ignored.
Missing arguments are given a default value (0 for numeric arguments, NULL for string arguments).
Extra arguments are ignored.

Unlike the vxWorks shell, string arguments do not have to be enclosed in quotes unless they contain one or more of the space characters, in which case one of the quoting mechanisms described in the following section must be used.

\subsection{Environment variable expansion}

Lines of input not beginning with a comment character (\verb|#|) are searched for macro references in the form \$\{name\} or \$(name).
The documentation for the macLib facility (chapter 19) describes some possible syntax variations for macro references.
Such references are replaced with the value of the environment variable they name before any other processing takes place.
Macro expansion is recursive so, for example,

\index{epicsEnvSet}
\begin{verbatim}
epics> epicsEnvSet v1 \${v2}
epics> epicsEnvSet v2 \${v3}
epics> epicsEnvSet v3 somePV
epics> dbpr ${v1}
\end{verbatim}

will print information about the \verb|somePV| process variable - the \verb|${v1}| argument to the dbpr command expands to \verb|${v2}| which expands to \verb|${v3}| which expands to \verb|somePV|.
The backslashes in the definitions are needed to postpone the substitution of the following variables, which would otherwise be performed before the \verb|epicsEnvSet| command was run.

Macro references that appear inside single-quotes are not expanded.

\subsection{Quoting}

Quoting is used to remove the special meaning normally assigned to certain characters and can be used to include space or quote characters in arguments.
Quoting takes place after the macro expansion described above has been performed, and cannot be used to extend a command over more than one input line.

There are three quoting mechanisms: the backslash character, single quotes, and double quotes.
A backslash (\textbackslash{}) preserves the literal value of the following character.
Enclosing characters in single or double quotes preserves the literal value of each character (including backslashes) within the quotes.
A single quote may occur between double quotes and a double quote may occur between single quotes.
Note that commands called from the shell may perform additional unescaping and macro expansion on their argument strings.

\subsection{Command-line editing and history}

The IOC shell can use the readline or tecla library to obtain input from the console terminal. This provides full command-line editing as well as easy access to previous commands through the command-line history capabilties provided by these libraries.
For full details, refer to the readline or tecla library documentation.
Command and argument completion is not supported.

If neither the readline nor tecla library is used the only command-line editing and history capabilities will be those supplied by the underlying operating system.
The console keyboard driver in Windows, for example, provides its own command-line editing and history commands.
On vxWorks the ledLib command-line input library routines are used.

\subsection{Redirection}

The IOC shell recognizes a subset of UNIX shell I/O redirection operators.
The redirection operators may precede, or appear anywhere within, or follow a command.
Redirections are processed in the order they appear, from left to right.
Failure to open or create a file causes the redirection to fail and the command to be ignored.

Redirection of input causes the file whose name results from the  expansion  of \verb|filename| to be  opened for reading on file descriptor \verb|n|, or the standard input (file descriptor 0) if \verb|n| is not specified.
The general format for redirecting input is:

\begin{verbatim}
[n]<filename
\end{verbatim}

As a special case, the IOC shell recognizes a standard input redirection appearing by itself (i.e. with no command) as a request to read commands from \verb|filename| until an exit command or EOF is encountered.
The IOC shell then resumes reading commands from the current source.
Commands read from \verb|filename| are not added to the readline command history.
The level of nesting is limited only by the maximum number of files that can be open simultaneously.

Redirection of output causes the file whose name results from the expansion of \verb|filename| to be opened for writing on file descriptor \verb|n|, or the standard output (file descriptor 1) if \verb|n| is not specified.
If the file does not exist it is created; if it does exist it is truncated to zero size.
The general format for redirecting output is:

\begin{verbatim}
[n]>filename
\end{verbatim}

 The general format for appending output is:

\begin{verbatim}
[n]>>filename
\end{verbatim}

Redirection of output in this fashion causes the \verb|filename| to be opened for appending on file descriptor \verb|n|, or the standard output (file descriptor 1) if \verb|n| is not specified.
If the file does not exist it is created.

\subsection{Utility Commands}
\label{Utility Commands}
\index{IOC Shell!utility commands}

The IOC shell recognizes the following commands as well as the commands described in chapter 6 (Database Definition) and chapter 9 (IOC Test Facilities) among others.
The commands described in the sequencer documentation will also be recognized if the sequencer is included.

\index{epicsEnvSet}
\index{epicsEnvShow}
\index{epicsParamShow}
\index{iocLogInit}
\index{epicsThreadSleep}
\begin{center}
\begin{longtable}{p{1.5in}p{4.5in}}
Command & Description\\
\hline
help [command ...] & Display synopsis of specified commands.  Wild-card matching is applied so `help db*' displays a synopsis of all commands beginning with the letters `db'. With no arguments this displays a list of all commands.\\
\# & A `\#' as the first non-whitespace character on a line marks the beginning of a comment, which continues to the end of the line (however some versions of Base may require a space after the `\#' character to properly recognize it as a comment)\\
exit & Stop reading commands. When the top-level command interpreter encounters an exit command or end-of-file (EOF) it returns to its caller.\\
cd directory & Change working directory to directory.\\
pwd & Print the name of the working directory.\\
var [name [value]] & If both arguments are present, assign the value to the named variable.If only the name argument is present, print the current value of that variable.If neither argument is present, print the value of all variables registered with the shell.  Variables are registered in application database definitions using the variable keyword as described in Section6.9 on page104.\\
show [-level] [task ...] & Show information about specified tasks.  If no task arguments are present, show information on all tasks.  The level argument controls the amount of information printed.  The default level is 0.  The task arguments can be task names or task i.d. numbers.\\
system command\_string & Send command\_string to the system command interpreter for execution.  This command is present only if some application database definition file contains registrar(iocshSystemCommand) and if the system provides a suitable command interpreter (vxWorks does not).\\
epicsEnvSet name value & Set environment variable name to the specified value.\\
epicsEnvShow  [name] & If no name is specified the names and values of all environment variables will be shown. If a name is specified the value of that environment variable will be shown.\\
epicsParamShow & Show names and values of all EPICS configuration parameters.\\
iocLogInit & Initialize IOC logging.\\
epicsThreadSleep sec & Pause execution of IOC shell for sec seconds.
\end{longtable}

\end{center}

The \verb|var| command is intended for simple applications such as setting the value of debugging flags.
Applications which require more complex expression handling should use the cexp package.

A \index{spy}\verb|spy| command to show periodic activity reports is available on RTEMS as part of the RTEMS\_UTILS support module.
The following changes must be made to add this command to an application.

\begin{itemize}
\item Add an RTEMS\_UTILS entry to the application configure/RELEASE file.

\item Add \verb|spy.dbd| to the list of application dbd files and \verb|rtemsutils| to the list of application libraries in the application Makefile.

\end{itemize}

\subsection{Environment Variables}
\index{IOC Shell!environment variables}

The IOC shell uses the following environment variables to control its operation.
\begin{center}
\begin{longtable}{p{1.35in}p{4.75in}}
Variable & Description\\
\hline
IOCSH\_PS1 & Prompt string. Default is ``epics\textgreater{} ''.\\
IOCSH\_HISTSIZE & Number of previous command lines to remember.
If the IOCSH\_HISTSIZE environment variable is not present the value of the HISTSIZE environment variable is used.
In the absence of both environment variables, 10 command lines will be remembered.\\
TERM, INPUTRC & These and other environment variables are used by the readline and termcap libraries and are described in the documentation for those libraries.
\end{longtable}

\end{center}

\subsection{Conditionals}
\index{Conditionals}
\index{IOC Shell!conditionals}

The IOC shell does not provide operaters for conditionally executing commands
but the effect can be simulated using macro expansion.  The simplest
technique is to preceed a command with a macro that expands to either `\verb@#@' or `' (or~`~').
The following startup script line shows how this can be done:
\begin{verbatim}
...
$(LOAD_DEBUG=#) $(DEBUG) dbLoadRecords("db/debugRec.db", "P=$(P),R=debug")
...
\end{verbatim}
Starting the IOC in the normal fashion will result in the above line being
commented out and the debugRec.db file being omitted:
\begin{verbatim}
./st.cmd
\end{verbatim}
Setting the \verb@LOAD_DEBUG@ environment variable to an empty string before
starting the IOC will result in the debugRec.db file being loaded:
\begin{verbatim}
LOAD_DEBUG="" ./st.cmd
\end{verbatim}

A similar technique can be used to execute external scripts conditionally.
The startup command file contains code like:
\begin{verbatim}
epicsEnvSet PILATUS_ENABLED "$(PILATUS_ENABLED=NO)"
...
< pilatus-$(PILATUS_ENABLED).cmd
\end{verbatim}
with one set of conditional code in a file named pilatus-YES.cmd and
the other set of conditional code in a file named pilatus-NO.cmd
This technique can be expanded to a form similar to a C `switch' statement
for the example above by providing additional pilatus-{\it XXX}.cmd scripts.


\section{IOC Shell Programming}

The declarations described in this section are included in the \verb|iocsh.h| header file.
\index{iocsh.h}

\subsection{Invoking the IOC shell}
\index{IOC Shell!invoking}

The prototypes for calling the IOC shell command interpreter are:

\begin{verbatim}
int iocsh(const char *pathname);
int iocshLoad(const char *pathname, const char *macros);
int iocshCmd(const char *cmd);
int iocshRun(const char *cmd, const char *macros);
\end{verbatim}

\index{iocsh}
\index{iocshLoad}
\index{iocshCmd}
\index{iocshRun}

The pathname argument to the \verb|iocsh| function is the name of the file from which commands are to be read.
If the pathname argument is NULL, commands are read from the standard input and prompts are issued to the standard output.
Commands are read until an \verb|exit| command is encountered or until end-of-file is reached, at which point iocsh returns a value of 0.
If the specified file can not be opened iocsh returns -1.

The IOC shell can be invoked from the vxWorks shell, either from within a vxWorks startup script or from vxWorks command-line interpreter, using

\begin{verbatim}
iocsh "script"
\end{verbatim}

to read from an IOC shell script.
It can also be invoked from the vxWorks command-line interpreter with no argument, in which case the IOC shell takes over the duties of command-line interaction.
The \verb|iocshLoad| function is an extension of the \verb|iocsh| command that takes an additional string consisting of a set of macro definitions.
This invokation of the IOC shell will then treat these macros as additional environment variables during execution, but will not persist after the shell exits.

The \verb|iocshCmd| function takes a single IOC shell command and executes it.
The \verb|iocshRun| function executes the command with additional macro replacement defined by the user in the second parameter.
These functions may be called from any thread, but many of the commands are not necessarily thread-safe so this should only be used with care.
These functions are most useful to execute a single IOC shell command from a vxWorks startup script or command line, like this:

\begin{verbatim}
iocshCmd "iocsh command string"
iocshRun "iocsh command string" "VAR=VAL"
\end{verbatim}

The stdio stream redirection and environment variable expansion processes described above are performed on the string as part of the execution process.

\subsection{Registering Commands}
\index{IOC Shell!registering commands}

Commands must be registered before they can be recognized by the IOC shell.
Registration is achieved by calling the registration function:

\begin{verbatim}
void iocshRegister(const iocshFuncDef *piocshFuncDef, iocshCallFunc func);
\end{verbatim}

\index{iocshRegister}
The first argument is a pointer to a data structure which describes the command and any arguments it may take.
The second argument is a pointer to a function which will be called by iocsh when the corresponding command is encountered.

The command is described by the  \verb|iocshFuncDef |structure:

\begin{verbatim}
struct iocshFuncDef {
    const char *name;
    int nargs;
    const iocshArg * const *arg;
};
\end{verbatim}

\index{iocshFuncDef}
The name element is the name of the command.
The arg element is a pointer to an array of pointers to structures each of which defines a single argument.
The nargs element declares the number of entries in the array of pointers to the argument descriptions.
If nargs is zero, arg can be NULL.
The structures which define each of the arguments is:

\begin{verbatim}
struct iocshArg {
    const char *name;
    iocshArgType type;
}iocshArg;
\end{verbatim}

\index{iocshArg}
The name element is used by the help command to print a synopsis for the command.
The type element describes the type of the argument and takes one of the following values:

\begin{center}
\begin{longtable}
{p{1.5in}p{3.76in}}
Type Specifier & Description\\
\hline
iocshArgInt & The argument will be converted to an integer value.\\
iocshArgDouble & The argument will be converted to a double-precision floating point value.\\
iocshArgString & The argument will be left as a string.  The memory used to hold the string is `owned' by iocsh and will be reused once the handler function returns.\\
iocshArgPersistentString & A copy of the argument will be made and a pointer to the copy will be passed to the handler.  The called function can release this copy by using the pointer as an argument to free().\\
iocshArgPdbbase & The argument must be pdbbase.\\
iocshArgArgv & An arbitrary number of arguments is expected.  Subsequent iocshArg structures will be ignored.
\end{longtable}
\end{center}


The `handler' function which is called when its corresponding command is recognized should be of the form:

\begin{verbatim}
void showCallFunc(const iocshArgBuf *args);
\end{verbatim}

The argument to the handler function is a pointer to an array of unions.
The number of elements in this array is equal to the number of arguments specified in the structure describing the command.
The type and name of the union element which contains the argument value depends on the `type' element of the corresponding argument descriptor:

\begin{center}
\begin{longtable}
{p{1.45833in}p{0.56in}p{1.19in}}
Type Specifier & Type & Union element\\
\hline
iocshArgInt & int & args[i].ival\\
iocshArgDouble & double & args[i].dval\\
iocshArgString  & char * & args[i].sval\\
iocshArgPersistentString & char * & args[i].sval\\
iocshArgPdbbase & void * & args[i].vval\\
iocshArgArgv & int & args[i].aval.ac \\
 & char ** & args[i].aval.av
\end{longtable}
\end{center}

If an \verb|iocshArgArgv |argument type is present it is often the first and only argument specified for the command.
In this case, \verb|args[0].aval.av[0]| will be the name of the command, \verb|args[0].aval.av[1]| will be the first argument, and so on.

\subsection{Registrar Command Registration}

\index{registrar!iocsh commands}
Commands are normally registered with the IOC shell in a registrar function.
The application's database description file uses the \verb|registrar| keyword to specify a function which will be called from the EPICS initialization code during the application startup process.
This function then calls \verb|iocshRegister| to register its commands with the iocsh.

\index{registrar}
The following code fragments shows how this can be performed for an example driver.

\begin{verbatim}
#include <iocsh.h>
#include <epicsExport.h>

/* drvXxx code, FuncDef and CallFunc definitions ... */

static void drvXxxRegistrar(void)
{
    iocshRegister(&drvXxxConfigureFuncDef, drvXxxConfigureCallFunc);
}
epicsExportRegistrar(drvXxxRegistrar);
\end{verbatim}

To include this driver in an application a developer would then add

\begin{verbatim}
registrar(drvXxxRegistrar)
\end{verbatim}

to an application database description file.

\subsection{Automatic Command Registration}

A C++ static constructor can also be used to register IOC shell commands before the EPICS application begins.
The following example shows how the \verb|epicsThreadSleep| command could be described and registered.

\begin{verbatim}
#include <iocsh.h>

static const iocshArg epicsThreadSleepArg0 = { "seconds",iocshArgDouble};
static const iocshArg *const epicsThreadSleepArgs[1] =
    {&epicsThreadSleepArg0};
static const iocshFuncDef epicsThreadSleepFuncDef =
    {"epicsThreadSleep",1,epicsThreadSleepArgs};
static void epicsThreadSleepCallFunc(const iocshArgBuf *args)
{
    epicsThreadSleep(args[0].dval);
}

static int doRegister(void)
{
    iocshRegister(epicsThreadSleepFuncDef, epicsThreadSleepCallFunc);
    return 1;
}
static int done = doRegister();
\end{verbatim}

