\chapter{Record Support}
\index{Record Support}

\section{Overview}

The purpose of this chapter is to describe record support in sufficient detail such that a C programmer can write new record support modules.
Before attempting to write new support modules, you should carefully study a few of the existing support modules.
If an existing support module is similar to the desired module most of the work will already be done.

From previous chapters, it should be clear that many things happen as a result of record processing.
The details of what happens are dependent on the record type.
In order to allow new record types and new device types without impacting the core IOC system, the concept of record support and device support is used.
For each record type, a record support module exists.
It is responsible for all record specific details.
In order to allow a record support module to be independent of device specific details, the concept of device support has been created.

A record support module consists of a standard set of routines which are called by database access routines.
These routines implement record specific code.
Each record type can define a standard set of device support routines specific to that record type.

By far the most important record support routine is \verb|process|, which \verb|dbProcess| calls when it wants to process a record.
This routine is responsible for the details of record processing.
In many cases it calls a device support I/O routine.
The next section gives an overview of what must be done in order to process a record.
Next is a description of the entry tables that must be provided by record and device support modules.
The remaining sections give example record and device support modules and describe some global routines useful to record support modules.

The record and its device support modules are the only source files that should include the record specific header files.
Thus they will be the only routines that access record specific fields without going through database access.

\section{Overview of Record Processing}

\index{Overview of Record Processing}
The most important record support routine is \verb|process|.
This routine determines what record processing means.
Before the record specific ``\verb|process|" routine is called, the following has already been done:

\begin{itemize}
\item Decision to process a record.

\item Check that record is not active, i.e. \verb|pact| must be FALSE.

\item Check that the record is not disabled.

\end{itemize}

The \verb|process| routine, together with its associated device support, is responsible for the following tasks:

\begin{itemize}
\item Set record active while it is being processed

\item Perform I/O (with aid of device support)

\item Check for record specific alarm conditions

\item Raise database monitors

\item Request processing of forward links

\end{itemize}

A complication of record processing is that some devices are intrinsically asynchronous.
It is NEVER permissible to wait for a slow device to complete.
Asynchronous records perform the following steps:

\begin{enumerate}
\item Initiate the I/O operation and set \verb|pact = TRUE|

\item Determine a method for again calling process when the operation completes

\item Return immediately without completing record processing

\item When process is called after the I/O operation complete record processing

\item Set \verb|pact = FALSE| and return

\end{enumerate}

The examples given below show how this can be done.

\section{Record Support and Device Support Entry Tables}

Each record type has an associated set of record support routines.
These routines are located via the \verb|struct rset| data structure declared in \verb|recSup.h| and defined by the specific record type.
This use of a record support vector table isolates the \verb|iocCore| software from the implementation details of each record type.
Thus new record types can be defined without having to modify the IOC core software.

Each record type also has zero or more sets of device support routines.
Record types without associated hardware, e.g. calculation records, normally do not have any associated device support.
Record types with associated hardware normally have a  device support module for each device type.
The concept of device support isolates IOC core software and even record support from device specific details.

Corresponding to each record type is a set of record support routines.
The set of routines is the same for every record type.
These routines are located via a Record Support Entry Table (\index{RSET}\index{Record Support Entry Table}RSET), which has the following structure:

\begin{verbatim}
struct rset {   /* record support entry table */
    long     number;            /* number of support routine */
    RECSUPFUN   report;         /* print report */
    RECSUPFUN   init;           /* init support */
    RECSUPFUN   init_record;    /* init record */
    RECSUPFUN   process;        /* process record */
    RECSUPFUN   special;        /* special processing */
    RECSUPFUN   get_value;      /* OBSOLETE: Just leave NULL */
    RECSUPFUN   cvt_dbaddr;     /* cvt  dbAddr */
    RECSUPFUN   get_array_info;
    RECSUPFUN   put_array_info;
    RECSUPFUN   get_units;
    RECSUPFUN   get_precision;
    RECSUPFUN   get_enum_str;   /* get string from enum */
    RECSUPFUN   get_enum_strs;  /* get all enum strings */
    RECSUPFUN   put_enum_str;   /* put enum from string */
    RECSUPFUN   get_graphic_double;
    RECSUPFUN   get_control_double;
    RECSUPFUN   get_alarm_double;
};
\end{verbatim}

Each record support module must define its RSET. The external name must be of the form:

\begin{verbatim}
<record_type>RSET
\end{verbatim}

Any routines not needed for the particular record type should be initialized to the value \verb|NULL|.
Look at the example below for details.

Device support routines are located via a Device Support Entry Table (\index{DSET}\index{Device Support Entry Table}DSET), which has the following structure:

\begin{verbatim}
struct dset {   /* device support entry table */
    long        number;     /* number of support routines */
    DEVSUPFUN   report;     /* print report */
    DEVSUPFUN   init;       /* init support */
    DEVSUPFUN   init_record;/* init record instance*/
    DEVSUPFUN    get_ioint_info;   /* get io interrupt info*/
    /* other functions are record dependent*/
};
\end{verbatim}

Each device support module must define its associated DSET.
The external name must be the same as the name which appears in \verb|devSup.ascii|.

Any record support module which has associated device support must also include definitions for accessing its associated device support modules.
The field \verb|dset|, which is declared in \verb|dbCommon|, contains the address of the DSET.
It is given a value by \verb|iocInit|.

\section{Example Record Support Module}

This section contains the skeleton of a record support package.
The record type is \verb|xxx| and the record has the following fields in addition to the \verb|dbCommon| fields:
\verb|VAL|, \verb|PREC|, \verb|EGU|, \verb|HOPR|, \verb|LOPR|, \verb|HIHI|, \verb|LOLO|, \verb|HIGH|, \verb|LOW|, \verb|HHSV|, \verb|LLSV|, \verb|HSV|, \verb|LSV|, \verb|HYST|, \verb|ADEL|, \verb|MDEL|, \verb|LALM|, \verb|ALST|, \verb|MLST|.
These fields will have the same meaning as they have for the \verb|ai| record.
Consult the Record Reference manual for a description.

\subsection{Declarations}

\begin{verbatim}
/* Create RSET - Record Support Entry Table*/
#define report NULL
#define initialize NULL
static long init_record();
static long process();
#define special NULL
#define get_value NULL
#define cvt_dbaddr NULL
#define get_array_info NULL
#define put_array_info NULL
static long get_units();
static long get_precision();
#define get_enum_str NULL
#define get_enum_strs NULL
#define put_enum_str NULL
static long get_graphic_double();
static long get_control_double();
static long get_alarm_double();

rset xxxRSET={
    RSETNUMBER,
    report,
    initialize,
    init_record,
    process,
    special,
    get_value,
    cvt_dbaddr,
    get_array_info,
    put_array_info,
    get_units,
    get_precision,
    get_enum_str,
    get_enum_strs,
    put_enum_str,
    get_graphic_double,
    get_control_double,
    get_alarm_double
};
epicsExportAddress(rset,xxxRSET);

/* declarations for associated DSET */
typedef struct xxxdset { /* analog input dset */
    long   number;
    DEVSUPFUN   dev_report;
    DEVSUPFUN   init;
    DEVSUPFUN   init_record; /* returns: (1,0)=> (failure, success)*/
    DEVSUPFUN   get_ioint_info;
    DEVSUPFUN   read_xxx;
}xxxdset;

/* forward declaration for internal routines*/
static void checkAlarams(xxxRecord *pxxx);
static void monitor(xxxRecord *pxxx);
\end{verbatim}

\index{RSET - example}The above declarations define the Record Support Entry Table (RSET), a template for the associated Device Support Entry Table (DSET), and forward declarations to private routines.

The RSET must be declared with an external name of \verb|xxxRSET|. It defines the record support routines supplied for this record type.
Note that forward declarations are given for all routines supported and a \verb|NULL| declaration for any routine not supported.

The template for the DSET is declared for use by this module.

\subsection{init\_record}

\index{init\_record - example}
\begin{verbatim}
static long init_record(void *precord, int  pass)
{
    xxxRecord*pxxx = (xxxRecord *)precord;
    xxxdset*pdset;
    longstatus;

    if(pass==0) return(0); 

    if((pdset = (xxxdset *)(pxxx->dset)) == NULL) {
        recGblRecordError(S_dev_noDSET,pxxx,"xxx: init_record");
        return(S_dev_noDSET);
    }
    /* must have read_xxx function defined */
    if( (pdset->number < 5) || (pdset->read_xxx == NULL) ) {
        recGblRecordError(S_dev_missingSup,pxxx,
        "xxx: init_record");
        return(S_dev_missingSup);
    }
    if( pdset->init_record ) {
        if((status=(*pdset->init_record)(pxxx))) return(status);
    }
    return(0);
}
\end{verbatim}

This routine, which is called by \verb|iocInit| twice for each record of type \verb|xxx|, checks to see if it has a proper set of device support routines and, if present, calls the \verb|init_record| entry of the DSET.

During the first call to \verb|init_record| (pass=0) only initializations relating to this record can be performed.
During the second call (pass=1) initializations that may refer to other records can be performed.
Note also that during the second pass, other records may refer to fields within this record.
A good example of where these rules are important is a waveform record.
The \verb|VAL| field of a waveform record actually refers to an array.
The waveform record support module must allocate storage for the array.
If another record has a database link referring to the waveform \verb|VAL| field then the storage must be allocated before the link is resolved.
This is accomplished by having the waveform record support allocate the array during the first pass (pass=0) and having the link reference resolved during the second pass (pass=1).

\subsection{process}

\index{process - example}
\begin{verbatim}
static long process(void *precord)
{
    xxxRecord*pxxx = (xxxRecord *)precord;
        xxxdset*pdset = (xxxdset *)pxxx->dset;
    longstatus;
    unsigned char pact=pxxx->pact;

    if( (pdset==NULL) || (pdset->read_xxx==NULL) ) {
        /* leave pact true so that dbProcess doesnt call again*/
        pxxx->pact=TRUE;
        recGblRecordError(S_dev_missingSup,pxxx,"read_xxx");
        return(S_dev_missingSup);
    }

    /* pact must not be set true until read_xxx completes*/
    status=(*pdset->read_xxx)(pxxx); /* read the new value */
     /* return if beginning of asynch processing*/
    if(!pact && pxxx->pact) return(0);
    pxxx->pact = TRUE;
    recGblGetTimeStamp(pxxx);

    /* check for alarms */
    alarm(pxxx);
    /* check event list */
    monitor(pxxx);
    /* process the forward scan link record */
    recGblFwdLink(pxxx);

    pxxx->pact=FALSE;
    return(status);
}
\end{verbatim}

The record processing routines are the heart of the IOC software.
The record specific process routine is called by \verb|dbProcess| whenever it decides that a record should be processed.
Process decides what record processing really means.
The above is a good example of what should be done.
In addition to being called by \verb|dbProcess| the process routine may also be called by asynchronous record completion routines.

The above model supports both synchronous and asynchronous device support routines.
For example, if \verb|read_xxx| is an asynchronous routine, the following sequence of events will occur:

\begin{itemize}
\item \verb|process| is called with \verb|pact| \verb|FALSE|

\item \verb|read_xxx| is called.
Since \verb|pact| is \verb|FALSE| it starts I/O, arranges callback, and sets \verb|pact| \verb|TRUE|

\item \verb|read_xxx| returns

\item because \verb|pact| went from \verb|FALSE| to \verb|TRUE| process just returns

\item Any new call to \verb|dbProcess| is ignored because it finds \verb|pact| \verb|TRUE|

\item Sometime later the callback occurs and \verb|process| is called again.

\item \verb|read_xxx| is called.
Since \verb|pact| is \verb|TRUE| it knows that it is a completion request.

\item \verb|read_xxx| returns

\item \verb|process| completes record processing

\item \verb|pact| is set \verb|FALSE|

\item \verb|process| returns

\end{itemize}

At this point the record has been completely processed.
The next time \verb|process| is called everything starts all over from the beginning.

\subsection{Miscellaneous Utility Routines}

\begin{verbatim}
static long get_units(DBADDR *paddr, char *units)
{
    xxxRecord  *pxxx=(xxxRecord *)paddr->precord;

    strncpy(units,pxxx->egu,sizeof(pxxx->egu));
    return(0);
}

static long get_graphic_double(DBADDR *paddr,
    struct dbr_grDouble *pgd)
{
    xxxRecord  *pxxx=(xxxRecord *)paddr->precord;
    int   fieldIndex = dbGetFieldIndex(paddr);

    if(fieldIndex == xxxRecordVAL) {
        pgd->upper_disp_limit = pxxx->hopr;
        pgd->lower_disp_limit = pxxx->lopr;
    } else recGblGetGraphicDouble(paddr,pgd);
    return(0);
}
/* similar routines would be provided for */
/* get_control_double and get_alarm_double*/
\end{verbatim}

\index{get\_units - .example}
\index{get\_graphic\_double - example}
These are a few examples of various routines supplied by a typical record support package.
The functions that must be performed by the remaining routines are described in the next section.

\subsection{Alarm Processing}

\begin{verbatim}
static void checkAlarms(xxxRecord *pxxx)
{
    double        val;
    float         hyst,lalm,hihi,high,low,lolo;
    unsigned short hhsv,llsv,hsv,lsv;

    if(pxxx->udf == TRUE ){
        recGblSetSevr(pxxx,UDF_ALARM,VALID_ALARM);
        return;
    }

    hihi=pxxx->hihi; lolo=pxxx->lolo;
    high=pxxx->high; low=pxxx->low;
    hhsv=pxxx->hhsv; llsv=pxxx->llsv;
    hsv=pxxx->hsv; lsv=pxxx->lsv;
    val=pxxx->val; hyst=pxxx->hyst; lalm=pxxx->lalm;

    /* alarm condition hihi */
    if (hhsv && (val >= hihi 
    || ((lalm==hihi) && (val >= hihi-hyst)))) {
        if(recGblSetSevr(pxxx,HIHI_ALARM,pxxx->hhsv)
            pxxx->lalm = hihi;
        return;
    }
    /* alarm condition lolo */
    if (llsv && (val <= lolo 
    || ((lalm==lolo) && (val <= lolo+hyst)))) {
        if(recGblSetSevr(pxxx,LOLO_ALARM,pxxx->llsv))
            pxxx->lalm = lolo;
        return;
    }
    /* alarm condition high */
    if (hsv && (val >= high 
    || ((lalm==high) && (val >= high-hyst)))) {
        if(recGblSetSevr(pxxx,HIGH_ALARM,pxxx->hsv))
            pxxx->lalm = high;
        return;
    }
    /* alarm condition low */
    if (lsv && (val <= low 
    || (lalm==low) && (val <= low+hyst)))) {
        if(recGblSetSevr(pxxx,LOW_ALARM,pxxx->lsv))
            pxxx->lalm = low;
    return;
    }
    /*we get here only if val is out of alarm by at least hyst*/
    pxxx->lalm=val;
    return;
}
\end{verbatim}

\index{checkAlarms}
This is a typical set of code for checking alarms conditions for an analog type record.
The actual set of code can be very record specific.
Note also that other parts of the system can raise alarms.
The algorithm is to always maximize alarm severity, i.e. the highest severity outstanding alarm will be reported.

The above algorithm also honors a hysteresis factor for the alarm.
This is to prevent alarm storms from occurring in the event that the current value is very near an alarm limit and noise makes it continually cross the limit.
It honors the hysteresis only when the value is going to a lower alarm severity.

Note the test:

\begin{verbatim}
if(pxxx->udf == TRUE ){
    recGblSetSevr(pxxx,UDF_ALARM,VALID_ALARM);
    return;
}
\end{verbatim}

\index{udf}
Database common defines the field \index{UDF}UDF, which means that field VAL is undefined.
The STAT and SEVR fields are initialized as though \verb|recGblSetSevr(pxxx,UDF_ALARM,VALID_ALARM)|was called.
Thus if the record is never processed the record will be in an INVALID UNDEFINED alarm state.
Field UDF is initialized to the value 1, i.e. TRUE.
Thus the above code will keep the record in the INVALID UNDEFINED alarm state as long as UDF is not given the value 0.

The UDF field means Undefined, i.e. the VAL field has never been given a value.
When records are loaded into an ioc this is the initial state of records.
Whevever code gives a value to the VAL field it is also supposed to set UDF false.
Unless a particular record type has unusual semantics no code should set UDF true.
UDF normally means that the field was never given a value.

For input records device support is responsible for obtaining an input value.
If no input value can be obtained neither record support nor device support sets UDF false.
If device support reads a raw value it returns a value telling record support to perform a conversion.
After the record support sets VAL equal to the converted value, it sets UDF false.
If device support obtains a converted value that it writes to VAL, it sets UDF false.

For output records either something outside record/device support writes to the VAL field or else VAL is given a value because record support obtains a value via the OMSL field.
In either case the code that writes to the VAL field sets UDF false.

Whenever database access writes to the VAL field it sets UDF false.

Routine recGblSetSevr is called to raise alarms.
It can be called by iocCore, record support, or device support.
The code that detects an alarm is responsible for raising the alarm.

\subsection{Raising Monitors}

\begin{verbatim}
static void monitor(xxxRecord *pxxx)
{
    unsigned short   monitor_mask;
    float            delta;

    monitor_mask = recGblResetAlarms(pxxx);
    /* check for value change */
    delta = pxxx->mlst - pxxx->val;
    if(delta<0.0) delta = -delta;
    if (delta > pxxx->mdel) {
        /* post events for value change */
        monitor_mask |= DBE_VALUE;
        /* update last value monitored */
        pxxx->mlst = pxxx->val;
    }
    /* check for archive change */
    delta = pxxx->alst - pxxx->val;
    if(delta<0.0) delta = 0.0;
    if (delta > pxxx->adel) {
        /* post events on value field for archive change */
        monitor_mask |= DBE_LOG;
        /* update last archive value monitored */
        pxxx->alst = pxxx->val;
    }
    /* send out monitors connected to the value field */
    if (monitor_mask){
        db_post_events(pxxx,&pxxx->val,monitor_mask);
    }
    return;
}
\end{verbatim}

\index{monitor - example}All record types should call \verb|recGblResetAlarms| as shown.
Note that \verb|nsta| and \verb|nsev| will have the value 0 after this routine completes.
This is necessary to ensure that alarm checking starts fresh after processing completes.
The code also takes care of raising alarm monitors when a record changes from an alarm state to the no alarm state.
It is essential that record support routines follow the above model or else alarm processing will not follow the rules.

Analog type records should also provide monitor and archive hysteresis fields as shown by this example.

\verb|db_post_events| results in channel access issuing monitors for clients attached to the record and field.
The call is

\begin{verbatim}
int db_post_events(void *precord, void *pfield,
    unsigned int monitor_mask)
\end{verbatim}
where:
\begin{description}
\item \verb|precord| - The address of the record

\item \verb|pfield| - The address of the field

\item \verb|monitor_mask| - A bit mask that can be any combinations of the following:


\begin{description}

\item \index{DBE\_ALARM}DBE\_ALARM - A change of alarm state has occured.
This is set by \verb|recGblResetAlarms|.

\item \index{DBE\_LOG}DBE\_LOG - Archive change of state.

\item \index{DBE\_VAL}DBE\_VAL - Value change of state

\end{description}

\end{description}
IMPORTANT:
The record support module is responsible for calling \verb|db_post_event| for any fields that change as a result of record processing.
Also it should NOT call \verb|db_post_event| for fields that do not change.

\section{Record Support Routines}

This section describes the routines defined in the RSET.
Any routine that does not apply to a specific record type must be declared \verb|NULL|.

\subsection{Generate Report of Each Field in Record}

\begin{verbatim}
long report(void *precord);
\end{verbatim}

\index{report - Record Support Routine}This routine is not used by most record types.
Any action is record type specific.

\subsection{Initialize Record Processing}

\begin{verbatim}
long initialize(void);
\end{verbatim}

\index{init - Record Support Routine}This routine is called once at IOC initialization time.
Any action is record type specific.
Most record types do not need this routine.

\subsection{Initialize Specific Record}

\begin{verbatim}
long init_record(void *precord, int pass);
\end{verbatim}

\index{init\_record - record support routine}\verb|iocInit| calls this routine twice (pass=0 and pass=1) for each database record of the type handled by this routine.
It must perform the following functions:

\begin{itemize}
\item Check and/or issue initialization calls for the associated device support routines.

\item Perform any record type specific initialization.

\item During the first pass it can only perform initializations that affect the record referenced by precord.

\item During the second pass it can perform initializations that affect other records.

\end{itemize}

\subsection{Process Record}

\begin{verbatim}
long process(void *precord);
\end{verbatim}

\index{process - Record Support Routine}
This routine must follow the guidelines specified previously.

\subsection{Special Processing}

\begin{verbatim}
long special(struct dbAddr *paddr, int after);
\end{verbatim}

\index{special - Record Support Routine}
This routine implements the record type specific special processing for the field referred to by \verb|dbAddr|.
It is called twice when a field is written to from outside the record, once with after=0 before any changes are made to the field, and again with after=1 after the change has been made.
The routine can prevent any changes from being made by returning an error status from the first call (after=0).
File \verb|special.h| defines special types.
This routine is only called for user special fields, i.e. fields with \verb|SPC_xxx| \textgreater{}= 100.
A field is declared special in the ASCII record definition file.
New values should not by added to \verb|special.h|, instead use \verb|SPC_MOD|.

The database access routine, \verb|dbGetFieldIndex| can be used to determine which field is being modified.

\subsection{Get Value}

This routine is no longer used.
It should be left as a NULL procedure in the record support entry table.

\subsection{Convert dbAddr Definitions}

\begin{verbatim}
long cvt_dbaddr(struct dbAddr *paddr);
\end{verbatim}

\index{cvt\_dbaddr - Record Support Routine}
This routine is called by \verb|dbNameToAddr| if the field has special set equal to \verb|SPC_DBADDR|.
A typical use is when a field refers to an array.
This routine can change any combination of the \verb|dbAddr| fields:
\verb|no_elements|, \verb|field_type|, \verb|field_size|, \verb|special,pfield, |and\verb| dbr_type|.
For example if the \verb|VAL| field of a waveform record is passed to \verb|dbNameToAddr|, \verb|cvt_dbaddr| would change \verb|dbAddr| so that it refers to the actual array rather then \verb|VAL|.

The database access routine, \verb|dbGetFieldIndex| can be used to determine which field is being modified.

NOTES:

\begin{itemize}
\item Channel access calls \verb|db_name_to_addr|, which is part of old database access.
\verb|db_name_to_addr| calls \verb|dbNameToAddr|.
This is done when a client connects to the record.

\item no\_elements must be set to the maximum number of elements that will ever be stored in the array.

\end{itemize}

\subsection{Get Array Information}

\begin{verbatim}
long get_array_info(struct dbAddr *paddr,
    long *no_elements, long *offset);
\end{verbatim}

\index{get\_array\_info - Record Support Routine}
This routine returns the current number of elements and the offset of the first value of the specified array.
The offset field is meaningful if the array is actually a circular buffer.

The database access routine, \verb|dbGetFieldIndex| can be used to determine which field is being modified.
It is permissible for \verb|get_array_info| to change \verb|pfield|.
This feature can be used to implement double buffering.

\index{get\_array\_info - Record Support Routine}
When an array field is being written \verb|get_array_info| is called before the field values are changed.

\subsection{Put Array Information}

\begin{verbatim}
long put_array_info(struct dbAddr *paddr, long nNew);
\end{verbatim}

\index{put\_array\_info - Record Support Routine}
This routine is called after new values have been placed in the specified array.

The database access routine, \verb|dbGetFieldIndex| can be used to determine which field is being modified.

\subsection{Get Units}

\begin{verbatim}
long get_units(struct dbAddr *paddr, char *punits);
\end{verbatim}

\index{get\_units - Record Support Routine}
This routine sets units equal to the engineering units for the field.

The database access routine, \verb|dbGetFieldIndex| can be used to determine which field is being modified.

\subsection{Get Precision}

\begin{verbatim}
long get_precision(struct dbAddr *paddr, long *precision);
\end{verbatim}

\index{get\_precision - Record Support Routine}
This routine gets the precision, i.e.
number of decimal places, which should be used to convert the field value to an ASCII string.
\verb|recGblGetPrec| should be called for fields not directly related to the value field.

The database access routine, \verb|dbGetFieldIndex| can be used to determine which field is being modified.

\subsection{Get Enumerated String}

\begin{verbatim}
long get_enum_str(struct dbAddr *paddr, char *p);
\end{verbatim}

\index{get\_enum\_str - record Support Routine}
This routine sets \verb|*p| equal to the ASCII string for the field value.
The field must have type \verb|DBF_ENUM|.

Look at the code for the \verb|bi| or \verb|mbbi| records for examples.

The database access routine, \verb|dbGetFieldIndex| can be used to determine which field is being modified.

\subsection{Get Strings for Enumerated Field}

\begin{verbatim}
long get_enum_strs(struct dbAddr *paddr, struct dbr_enumStrs *p);
\end{verbatim}

\index{get\_enum\_strs - record Support Routine}
This routine gives values to all fields of structure \verb|dbr_enumStrs|.

Look at the code for the \verb|bi| or \verb|mbbi| records for examples.

The database access routine, \verb|dbGetFieldIndex| can be used to determine which field is being modified.

\subsection{Put Enumerated String}

\begin{verbatim}
long put_enum_str(struct dbAddr *paddr, char *p);
\end{verbatim}

\index{put\_enum\_str - Record Support Routine}
Given an ASCII string, this routine updates the database field.
It compares the string with the string values associated with each enumerated value and if it finds a match sets the database field equal to the index of the string which matched.

Look at the code for the \verb|bi| or \verb|mbbi| records for examples.

The database access routine, \verb|dbGetFieldIndex| can be used to determine which field is being modified.

\subsection{Get Graphic Double Information}

\begin{verbatim}
long get_graphic_double(struct dbAddr *paddr, struct dbr_grDouble *p);
\end{verbatim}

\index{get\_graphic\_double - Record Support Routine}
This routine fills in the graphics related fields of structure \verb|dbr_grDouble|.
\verb|recGblGetGraphicDouble| should be called for fields not directly related to the value field.

The database access routine, \verb|dbGetFieldIndex| can be used to determine which field is being modified.

\subsection{Get Control Double Information}

\begin{verbatim}
long get_control_double(struct dbAddr *paddr, struct dbr_ctrlDouble  *p);
\end{verbatim}

\index{get\_control\_double - Record Support Routine}
This routine gives values to all fields of structure \verb|dbr_ctrlDouble|.
\verb|recGblGetControlDouble| should be called for fields not directly related to the value field.

The database access routine, \verb|dbGetFieldIndex| can be used to determine which field is being modified.

\subsection{Get Alarm Double Information}

\begin{verbatim}
long get_alarm_double(struct dbAddr *paddr, struct dbr_alDouble *p);
\end{verbatim}

\index{get\_alarm\_double Record Support Routine}
This routine gives values to all fields of structure \verb|dbr_alDouble|.

The database access routine, \verb|dbGetFieldIndex| can be used to determine which field is being modified.

\section{Global Record Support Routines}

\index{Global Record Support Routines}
A number of global record support routines are available.
These routines are intended for use by the record specific processing routines but can be called by any routine that wishes to use their services.

The name of each of these routines begins with ``\verb|recGbl|".
Code using these routines should
\begin{verbatim}
#include <recGbl.h>
\end{verbatim}

\subsection{Alarm Status and Severity}

Alarms may be raised in many different places during the course of record processing.
The algorithm is to maximize the alarm severity, i.e. the highest severity outstanding alarm is raised.
If more than one alarm of the same severity is found then the first one is reported.
This means that whenever a code fragment wants to raise an alarm, it does so only if the alarm severity it will declare is greater then that already existing.
Four fields (in database common) are used to implement alarms:
\verb|sevr|, \verb|stat|, \verb|nsev|, and \verb|nsta|.
The first two are the status and severity after the record is completely processed.
The last two fields (\verb|nsta| and \verb|nsev|) are the status and severity values to set during record processing.
Two routines are used for handling alarms.
Whenever a routine wants to raise an alarm it calls \verb|recGblSetSevr|.
This routine will only change \verb|nsta| and \verb|nsev| if it will result in the alarm severity being increased.
At the end of processing, the record support module must call \verb|recGblResetAlarms|.
This routine sets \verb|stat = nsta|, \verb|sevr = nsev|, \verb|nsta= 0|, and \verb|nsev = 0|.
If \verb|stat| or \verb|sevr| has changed value since the last call it calls \verb|db_post_event| for \verb|stat| and \verb|sevr| and returns a value of \verb|DBE_ALARM|.
If no change occured it returns 0.
Thus after calling \verb|recGblResetAlarms| everything is ready for raising alarms the next time the record is processed.
The example record support module presented above shows how these macros are used.

\begin{verbatim}
int recGblSetSevr(void *precord,
    short nsta, short nsevr);
\end{verbatim}

\index{recGblSetSevr}
Returns \verb|TRUE| if it changed \verb|nsta| and/or \verb|nsev|, \verb|FALSE| if it did not change them.

\begin{verbatim}
unsigned short recGblResetAlarms(void  *precord);
\end{verbatim}

\index{recGblResetAlarms}
Returns: Initial value for \verb|monitor_mask|

\subsection{Alarm Acknowledgment}

Database common contains two additional alarm related fields:

\begin{itemize}
\item \verb|acks| - highest severity unacknowledged alarm
\item \verb|ackt| - do transient alarm need to be acknowledged
\end{itemize}

These fields are handled by \verb|iocCore| and \verb|recGblResetAlarms| and should not be used by record support.
The alarm acknowledgement facility it provided for use by alarm handlers.

\subsection{Generate Error: Process Variable Name, Caller, Message}

SUGGESTION: use \index{errlogPrintf}\verb|errlogPrintf| instead of this for new code.

\begin{verbatim}
void recGblDbaddrError(
    long   status,
    struct dbAddr  *paddr,
    char   *pcaller_name); /* calling routine name */
\end{verbatim}

\index{recGblDbaddrError}
This routine interfaces with the system wide error handling system to display the following information:
Status information, process variable name, calling routine.

\subsection{Generate Error: Status String, Record Name, Caller}
SUGGESTION: use errlogPrintf instead of this for new code.
\begin{verbatim}
void recGblRecordError(
    long   status,
    void   *precord,
    char   *pcaller_name);   /* calling routine name */
\end{verbatim}

\index{errlogPrintf}
\index{recGblRecordError}
This routine interfaces with the system wide error handling system to display the following information:
Status information, record name, calling routine.

\subsection{Generate Error: Record Name, Caller, Record Support Message}
SUGGESTION: use errlogPrintf instead of this for new code.
\begin{verbatim}
void recGblRecsupError(
    long   status,
    struct   dbAddr   *paddr,
    char   *pcaller_name,   /* calling routine name */
    char   *psupport_name);   /* support routine name*/
\end{verbatim}

\index{errlogPrintf}
\index{recGblRecsupError}
This routine interfaces with the system wide error handling system to display the following information:
Status information, record name, calling routine, record support entry name.

\subsection{Get Graphics Double}

\begin{verbatim}
void recGblGetGraphicDouble(struct dbAddr *paddr, struct dbr_grDouble *pgd);
\end{verbatim}

\index{recGblGetGraphicDouble}
This routine can be used by the \verb|get_graphic_double| record support routine to obtain graphics values for fields that it doesn't know how to set.

\subsection{Get Control Double}

\begin{verbatim}
void recGblGetControlDouble(struct dbAddr *paddr, struct dbr_ctrlDouble *pcd);
\end{verbatim}

\index{recGblGetControlDouble}
This routine can be used by the \verb|get_control_double| record support routine to obtain control values for fields that it doesn't know how to set.

\subsection{Get Alarm Double}

\begin{verbatim}
void recGblGetAlarmDouble(struct dbAddr *paddr, struct dbr_alDouble *pcd);
\end{verbatim}

\index{recGblGetAlarmDouble}
This routine can be used by the \verb|get_alarm_double| record support routine to obtain control values for fields that it doesn't know how to set.

\subsection{Get Precision}

\begin{verbatim}
void recGblGetPrec(struct dbAddr *paddr, long *pprecision);
\end{verbatim}

\index{recGblGetPrec}
This routine can be used by the \verb|get_precision| record support routine to obtain the precision for fields that it doesn't know how to set the precision.

\subsection{Get Time Stamp}

\begin{verbatim}
void recGblGetTimeStamp(void *precord)
\end{verbatim}

\index{recGblGetTimeStamp}
This routine gets the current time stamp and puts it in the record It does the following:

\begin{itemize}
\item If TSEL is not a constant link and TSEL refers to the TIME field of a record, the time is obtained from the record reference by TSEL and this put in field TIME.
The routine then returns.

\item If TSEL is not a constant link dbGetLink is called and the value put in field TSE.

\item If TSE is equal to \verb|epicsTimeEventDeviceTime| (-2) then noting is done, i.e. the routine just returns.

\item \verb|epicsTimeGetEvent| is called.

\end{itemize}

\subsection{Forward link}

\begin{verbatim}
void recGblFwdLink(void *precord);
\end{verbatim}

\index{recGblFwdLink}
This routine can be used by process to request processing of forward links.

\subsection{Initialize Constant Link}

\begin{verbatim}
int recGblInitConstantLink(
    struct link  *plink,
    short  dbfType,
    void   *pdest);
\end{verbatim}

\index{recGblInitConstantLink}
Initialize a constant link.
This routine is usually called by \verb|init_record| (or by associated device support) to initialize the field associated with a constant link.
It returns(FALSE, TRUE) if it (did not, did) modify the destination.

\subsection{Analog Value Deadband Check}

\begin{verbatim}
void recGblCheckDeadband(
    epicsFloat64 *poldval,
    const epicsFloat64 newval,
    const epicsFloat64 deadband,
    unsigned *monitor_mask,
    const unsigned add_mask);
\end{verbatim}

\index{recGblCheckDeadband}
Check if analog (double) value is outside specified deadband, and set bits in monitor mask.
This routine is usually called by an analog record's \verb|monitor| (as part of processing) to check if a value is outside a predefined deadband.
It also set bits in a monitor mask according to the check result.
If \verb|newval| lies outside the specified \verb|deadband|, \verb|newval| is copied into \verb|*poldval|, and \verb|add_mask| is OR'ed into \verb|monitor_mask|.
