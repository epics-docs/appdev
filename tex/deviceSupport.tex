\chapter{Device Support}
\index{Device Support}

\section{Overview}

In addition to a record support module, each record type can have an arbitrary number of device support modules. The 
purpose of device support is to hide hardware specific details from record processing routines. Thus support can be 
developed for a new device without changing the record support routines.

A device support routine has knowledge of the record definition. It also knows how to talk to the hardware directly or how 
to call a device driver which interfaces to the hardware. Thus device support routines are the interface between hardware 
specific fields in a database record and device drivers or the hardware itself.

Release 3.14.8 introduced the concept of \index{Extended device support}extended device support, which provides an optional interface that a device 
support can implement to obtain notification when a record's address is changed at runtime. This permits records to be 
reconnected to a different kind of I/O device, or just to a different signal on the same device.
Extended device support is described in more detail in Section \ref{sec:Extended Device Support} below.

Database common contains two device related fields:

\begin{itemize}
\item \index{dtyp - dbCommon}dtyp:  Device Type.

\item \index{dset - dbCommon}dset:  Address of Device Support Entry Table.

\end{itemize}

The field \verb|DTYP| contains the index of the menu choice as defined by the device ASCII definitions. \verb|iocInit| uses this 
field and the device support structures defined in \verb|devSup.h| to initialize the field \verb|DSET|. Thus record support can locate 
its associated device support via the \verb|DSET| field.

Device support modules can be divided into two basic classes: synchronous and asynchronous. Synchronous device 
support is used for hardware that can be accessed without delays for I/O. Many register based devices are synchronous 
devices. Other devices, for example all GPIB devices, can only be accessed via I/O requests that may take large amounts 
of time to complete. Such devices must have associated asynchronous device support. Asynchronous device support 
makes it more difficult to create databases that have linked records.

If a device can be accessed with a delay of less then a few microseconds then synchronous device support is appropriate. 
If a device causes delays of greater than 100 microseconds then asynchronous device support is appropriate. If the delay is 
between these values your guess about what to do is as good as mine. Perhaps you should ask the hardware designer why 
such a device was created.

If a device takes a long time to accept requests there is another option than asynchronous device support. A driver can be 
created that periodically polls all its attached input devices. The device support just returns the latest polled value. For 
outputs, device support just notifies the driver that a new value must be written. the driver, during one of its polling 
phases, writes the new value. The EPICS Allen Bradley device/driver support is a good example.

\section{Example Synchronous Device Support Module}

\begin{lstlisting}[language=C]
/* Create the dset for devAiSoft */
long init_record();
long read_ai();
struct {
    long   number;
    DEVSUPFUN   report;
    DEVSUPFUN   init;
    DEVSUPFUN   init_record;
    DEVSUPFUN   get_ioint_info;
    DEVSUPFUN   read_ai;
    DEVSUPFUN   special_linconv;
}devAiSoft={
    6,
    NULL,
    NULL,
    init_record,
    NULL,
    read_ai,
    NULL
};
epicsExportAddress(dset,devAiSoft);

static long init_record(void *precord)
{
    aiRecord  *pai = (aiRecord *)precord;
    long status;

    /* ai.inp must be a CONSTANT, PV_LINK, DB_LINK or CA_LINK*/
    switch (pai->inp.type) {
        case (CONSTANT) :
            if(recGblInitConstantLink(&pai->inp,DBF_DOUBLE,&pai->val))
                pai->udf = FALSE;
            break;

        case (PV_LINK) :
        case (DB_LINK) :
        case (CA_LINK) :
            break;
        default :
            recGblRecordError(S_db_badField, (void *)pai,
                "devAiSoft (init_record) Illegal INP field");
        return(S_db_badField);
    }
    /* Make sure record processing routine does not perform any conversion*/
    pai->linr=0;
    return(0);
}

static long read_ai(void *precord)
{
    aiRecord*pai =(aiRecord *)precord;
    long status;

    status=dbGetGetLink(&(pai->inp.value.db_link),
    (void *)pai,DBR_DOUBLE,&(pai->val),0,1);
    if (pai->inp.type!=CONSTANT && RTN_SUCCESS(status)) pai->udf = FALSE;
    return(2); /*don't convert*/
}
\end{lstlisting}

\index{synchronous device support example}
The example is \verb|devAiSoft|, which supports soft analog inputs.
The \verb|INP| field can be a constant or a database link or a channel access link.
Only two routines are provided (the rest are declared \verb|NULL|).
The \verb|init_record| routine first checks that the link type is valid.
If the link is a constant it initializes \verb|VAL|.
If the link is a Process Variable link it calls \verb|dbCaGetLink| to turn it into a Channel Access link.
The \verb|read_ai| routine obtains an input value if the link is a database 
or Channel Access link, otherwise it doesn't have to do anything.  

\section{Example Asynchronous Device Support Module}

\index{asynchronous device support example}
This example shows how to write an asynchronous device support routine.
It does the following sequence of operations:

\begin{enumerate}
\item When first called \verb|PACT| is \verb|FALSE|.
It arranges for a callback (\verb|myCallback|) routine to be called after a number of seconds specified by the \verb|DISV| field.

\item It prints a message stating that processing has started, sets \verb|PACT| to \verb|TRUE|, and returns.
The record processing routine returns without completing processing.

\item When the specified time elapses \verb|myCallback| is called.
It calls \verb|dbScanLock| to lock the record, calls \verb|process|, and calls \verb|dbScanUnlock| to unlock the record.
It directly calls the \verb|process| entry of the record support module, which it locates via the \verb|RSET| field in \verb|dbCommon|, rather than calling \verb|dbProcess|. \verb|dbProcess| would not call \verb|process| because \verb|PACT| is \verb|TRUE|. 

\item When \verb|process| executes, it again calls \verb|read_ai|.
This time \verb|PACT| is \verb|TRUE|.

\item \verb|read_ai| prints a message stating that record processing is complete and returns a status of 2.
Normally a value of  0 would be returned.
The value 2 tells the record support routine not to attempt any conversions.
This is a convention (a bad convention!) used by the analog input record.

\item When \verb|read_ai| returns the record processing routine completes record processing.

\end{enumerate}

At this point the record has been completely processed.
The next time process is called everything starts all over.

Note that this is somewhat of an artificial example since real code of this form would more likely use the \verb|callbackRequestProcessCallbackDelayed| function to perform the required processing.

\begin{lstlisting}[language=C]
static void myCallback(CALLBACK *pcallback)
{
    struct dbCommon   *precord;
    struct typed_rset *prset;

    callbackGetUser(precord,pcallback);
    prset = (struct typed_rset *)(precord->rset);
    dbScanLock(precord);
    (*prset->process)(precord);
    dbScanUnlock(precord);
}

static long init_record(struct aiRecord *pai)
{
    CALLBACK *pcallback;
    switch (pai->inp.type) {
    case (CONSTANT) :
        pcallback = (CALLBACK *)(calloc(1,sizeof(CALLBACK)));
        callbackSetCallback(myCallback,pcallback);
        callbackSetUser(pai,pcallback);
        pai->dpvt = (void *)pcallback;
        break;
    default :
        recGblRecordError(S_db_badField,(void *)pai,
            "devAiTestAsyn (init_record) Illegal INP field");
        return(S_db_badField);
    }
    return(0);
}

static long read_ai(struct aiRecord *pai)
{
    CALLBACK *pcallback = (CALLBACK *)pai->dpvt;
    if(pai->pact) {
        pai->val += 0.1; /* Change VAL just to show we've done something. */
        pai->udf = FALSE; /* We modify VAL so we are responsible for UDF too. */
        printf("Completed asynchronous processing: %s\n",pai->name);
        return(2); /* don't convert*/
    } 
    printf("Starting asynchronous processing: %s\n",pai->name);
    pai->pact=TRUE;
    callbackRequestDelayed(pcallback,pai->disv);
    return(0);
}

/* Create the dset for devAiTestAsyn */
struct {
    long        number;
    DEVSUPFUN   report;
    DEVSUPFUN   init;
    DEVSUPFUN   init_record;
    DEVSUPFUN   get_ioint_info;
    DEVSUPFUN   read_ai;
    DEVSUPFUN   special_linconv;
}devAiTestAsyn={
    6,
    NULL,
    NULL,
    init_record,
    NULL,
    read_ai,
    NULL
};
epicsExportAddress(dset,devAiTestAsyn);
\end{lstlisting}

\section{Device Support Routines}

This section describes the routines defined in the DSET.
Any routine that does not apply to a specific record type must be declared \verb|NULL|.

\subsection{Generate Device Report}

\begin{lstlisting}[language=C]
long report(
    int   interest);
\end{lstlisting}

\index{report - device support routine}
This routine is responsible for reporting all I/O cards it has found.
The \verb|interest| value is provided to allow for different kinds of reports, or to control how much detail to display.
If a device support module is using a driver, it may choose not to implement this routine because the driver generates the report.

\subsection{Initialize Device Processing}
\label{subsec:Initialize Device Processing}

\index{Initialize Device Processing}
\begin{lstlisting}[language=C]
long init(
    int   after);
\end{lstlisting}

\index{init - device support routine}
This routine is called twice at IOC initialization time.
Any action is device specific.
This routine is called twice: once before any database records are initialized, and once after all records are initialized but before the scan tasks are started.
\verb|after| has the value 0 before and 1 after record initialization.

\subsection{Initialize Specific Record}
\label{subsec:Initialize Specific Record}

\index{Initialize Specific Record}
\begin{lstlisting}[language=C]
long init_record(
    void *precord);   /* addr of record*/
\end{lstlisting}

\index{init\_record - device support routine}
The record support \verb|init_record| routine calls this routine.

\subsection{Get I/O Interrupt Information}

\begin{lstlisting}[language=C]
long get_ioint_info(
    int   cmd,
    struct dbCommon   *precord,
    IOSCANPVT   *ppvt);
\end{lstlisting}

\index{get\_ioint\_info - device support routine}
This is called by the I/O interrupt scan task.
If \verb|cmd| is (0,1) then this routine is being called when the associated record is being (placed in, taken out of) an I/O scan list.
See chapter \ref{chap:Database Scanning} for details.

\subsection{Other Device Support Routines}

All other device support routines are record type specific.

\section{Extended Device Support}
\label{sec:Extended Device Support}

\index{Extended device support}
This section describes the additional behaviour and routines required for a device support layer to support online changes to a record's hardware address.

\subsection{Rationale}

In releases prior to R3.14.8 it is possible to change the value of the INP or OUT field of a record but (unless a soft device support is in use) this generally has no effect on the behaviour of the device support at all.
Some device supports have been written that check this hardware address field for changes every time they process, but they are in the minority and in 
any case they do not provide any means to switch between different device support layers at runtime since no software is present that can lookup a new value for the DSET field after iocInit.

The extended device interface has been carefully designed to retain maximal backwards compatibility with existing device and record support layers, and as a result it cannot just introduce new routines into the DSET:

\begin{itemize}
\item Different record types have different numbers of DSET routines

\item Every device support layer defines its own DSET structure layout

\item Some device support layers add their own routines to the DSET (GPIB, BitBus)

\end{itemize}

Since both basic and extended device support layers have to co-exist within the same IOC, some rules are enforced concerning whether the device address of a particular record is allowed to be changed:

\begin{enumerate}
\item Records that were connected at iocInit to a device support layer that does not implement the extended interface are never allowed to have address fields changed at runtime.

\item Extended device support layers are not required to implement both the \verb|add_record| and \verb|del_record| routines, thus some devices may only allow one-way changes.

\item The old device support layer is informed and allowed to refuse an address change before the field change is made (it does not get to see the new address).

\item The new device support layer is informed after the field change has been made, and it may refuse to accept the record.
In this case the record will be set as permanently busy (PACT=true) until an address is accepted.

\item Record support layers can also get notified about this process by making their address field special, in which case the record type's special routine can refuse to accept the new address before it is presented to the device support layer.
Special cannot prevent the old device support from being disconnected however.

\end{enumerate}

If an address change is refused, the change to the INP or OUT field will cause an error or exception to be passed to the software performing the change.
If this was a Channel Access client the result is to generate an exception callback.

To switch to a different device support layer, it is necessary to change the DTYP field before the INP or OUT field.
The change to the DTYP field has no effect until the latter field change takes place.

If a record is set to I/O Interrupt scan but the new layer does not support this, the scan will be changed to Passive.

\subsection{Initialization/Registration}

Device support that implements the extended behaviour must provide an \verb|init| routine in the Device Support Entry Table 
(see Section \ref{Initialize Device Processing}).
In the first call to this routine (pass 0) it registers the address of its Device Support eXtension Table (DSXT) in a call to \verb|devExtend|.

The only exception to this registration requirement is when the device support uses a link type of CONSTANT.  In this 
circumstance the system will automatically register an empty DSXT for that particular support layer (both the 
\verb|add_record| and \verb|del_record| routines pointed to by this DSXT do nothing and return zero). This exception allows 
existing soft channel device support layers to continue to work without requiring any modification, since the iocCore 
software already takes care of changes to PV\_LINK addresses.

The following is an example of a DSXT and the initialization routine that registers it:

\begin{lstlisting}[language=C]
static struct dsxt myDsxt = {
    add_record, del_record
};

static long init(int pass) {
    if (pass==0) devExtend(&myDsxt);
    return 0;
}
\end{lstlisting}

A call to \verb|devExtend| can only be made during the first pass of the device support initialization process, and registers the DSXT for that device support layer.
If called at any other time it will log an error message and immediately return.

\subsection{Device Support eXtension Table}

The full definition of \verb|struct dsxt| is found in devSup.h and currently looks like this:

\begin{lstlisting}[language=C]
typedef struct dsxt {
    long (*add_record)(struct dbCommon *precord);
    long (*del_record)(struct dbCommon *precord);
} dsxt;
\end{lstlisting}

\index{struct dsxt}
There may be future additions to this table to support additional functionality; such extensions may only be made by changing the devSup.h header file and rebuilding EPICS Base and all support modules, thus neither record types nor device support are permitted to make any private use of this table.

The two function pointers are the means by which the extended device support is notified about the record instances it is being given or that are being moved away from its control.
In both cases the only parameter is a pointer to the record concerned, which the code will have to cast to the appropriate pointer for the record type.
The return value from the routines should be zero for success, or an EPICS error status code.

\subsection{Add Record Routine}

\begin{lstlisting}[language=C]
long add_record(
    struct dbCommon *precord);
\end{lstlisting}

\index{add\_record}
This function is called to offer a new record to the device support.
It is also called during iocInit, in between the pass 0 and pass 1 calls to the regular device support \verb|init_record| routine (described in Section \ref{subsec:Initialize Specific Record} above).
When converting an existing device support layer, this routine will usually be very similar to the old \verb|init_record| routine, although in some cases it may be necessary to do a little more work depending on the particular record type involved.
The extra code required in these cases can generally be copied straight from the record type implementation itself.
This is necessary because the record type has no knowledge of the address change that is taking place, so the device support must perform any bitmask generation and/or readback value conversions itself.
This document does not attempt to describe all the necessary processing for the various different standard record types, although the following (incomplete) list is presented as an aid to device support authors:

\begin{itemize}
\item mbbi/mbbo record types:
Set SHFT, convert NOBT and SHFT into MASK

\item bi/bo record types:
Set SHFT, convert SHFT to MASK

\item analog record types:
Calculate ESLO and EOFF

\item Output record types:
Possibly read the current value from hardware and back-convert to VAL, or send the current record output value to the hardware. \emph{This behaviour is not required or defined, and it's not obvious what should be done. There may be complications here with ao records using OROC and/or OIF=Incremental; solutions to this issue have yet to be considered by the community.}

\end{itemize}

If the \verb|add_record| routine discovers any errors, say in the link address, it should return a non-zero error status value to reject the record.
This will cause the record's PACT field to be set, preventing any further processing of this record until some other address change to it gets accepted.

\subsection{Delete Record Routine}

\begin{lstlisting}[language=C]
long del_record(
    struct dbCommon *precord);
\end{lstlisting}

\index{del\_record}
This function is called to notify the device support of a request to change the hardware address of a record, and allow the device support to free up any resources it may have dedicated to this particular record.

Before this routine is called, the record will have had its SCAN field changed to Passive if it had been set to I/O Interrupt.
This ensures that the device support's \verb|get_ioint_info| routine is never called after the the call to \verb|del_record| has returned successfully, although it may also lead to the possibility of missed interrupts if the address change is rejected by the \verb|del_record| routine.

If the device support is unable to disconnect from the hardware for some reason, this routine should return a non-zero error status value, which will prevent the hardware address from being changed.
In this event the SCAN field will be restored if it was originally set to I/O Interrupt.

After a successfull call to \verb|del_record|, the record's DPVT field is set to NULL and PACT is cleared, ready for use by the new device support.

\subsection{Init Record Routine}

The \verb|init_record| routine from the DSET (section \ref{subsec:Initialize Specific Record}) is called by the record type, and must still be provided since the record type's per-record initialization is run some time \emph{after} the initial call to the DSXT's \verb|add_record| routine.
Most record types perform some initialization of record fields at this point, and an extended device support layer may have to fix anything that the record overwrites.
The following (incomplete) list is presented as an aid to device support authors:

\begin{itemize}
\item mbbi/mbbo record types:
Calculate MASK from SHFT

\item analog record types:
Calculate ESLO and EOFF

\item Output record types:
Perform readback of the initial raw value from the hardware.

\end{itemize}

